\chapter{Estrutura do Domain Name System}

Desde o início das discussões sobre um espaço de nomes, sempre foi
cogitada uma solução hierárquica e distribuída \cite{rfc1034} para a
implementação.

Hoje, a base de dados do DNS é indexada por nomes de domínio. Cada
domínio é um caminho em uma árvore invertida. A estrutura de árvore é
similar à estrutura do sistema de arquivos utilizado nos sistemas
\textit{Unix}. A árvore tem uma única raiz, chamado de diretório
\textit{root} (raiz) no \textit{Unix}, e representado por uma barra
(/). No DNS, esse diretório é chamado simplesmente de ``raiz'', e não
recebe nome.

Antes de prosseguir com a estrutura, é necessário fazer algumas
definições sobre os termos usados neste documento.

\section{Gramática}

A sintaxe a seguir foi feita para evitar problemas de ambiguidade e
definições \cite{rfc1035}:

\begin{itemize}
\item <domínio> ::= <subdomínio> | " "
\item <subdomínio> ::= <rótulo> | <subdomínio> "." <rótulo>
\item <rótulo> ::= <letra> [ [ <string> ] <letra--dígito> ]
\item <string> ::= <letra--dígito--hífem> | <letra--dígito--hífem> <string>
\item <letra--dígito--hífem> ::= <letra--dígito> | "--"
\item <letra--dígito> ::= <letra> | <dígito>
\item <letra> ::= qualquer um dos 52 caracteres alfabéticos de
  \emph{A} a \emph{Z} maiúsculos e \emph{a} a \emph{z} minúsculos.
\item <dígito> ::= qualquer um dos dez dígitos de 0 a 9.
\end{itemize}

Note que, apesar de serem permitidas letras maiúsculas e minúsculas,
elas não tem significado diferente no DNS, de modo que o pseudo--domínio
a.b.c deve ser o mesmo que A.B.C.

Os rótulos devem seguir as regras da ARPANET para nomes de máquinas:
devem começar com uma letra e terminar com uma letra ou dígito, e ter
seu interior formado por letras, dígitos ou hífens. Rótulos devem ter
menos de 64 caracteres.

\section{Os elementos do DNS}

O DNS é formado por três elementos principais \cite{rfc1034}:

\begin{enumerate}
\item \textbf{Nomes de domínio} e \textbf{Registros}, que são as
  especificações para o espaço de nomes organizado em estrutura de
  árvore, e os dados associados aos nomes.

\item \textbf{Servidores de nomes}, que são programas de servidores que
  contêm informações sobre a estrutura do domínio de nomes. Um Servidor
  de nomes pode armazenar em cache a estrutura ou os registros de
  qualquer parte da árvore de domínio de nomes, mas geralmente um
  servidor de nomes em particular tem informação completa sobre um
  subconjunto do domínio de nomes, e ponteiros para outros servidores de
  nomes, que podem ser usados para obter informações sobre qualquer
  parte da árvore do domínio de nomes. Esses servidores que têm
  informação completa de certas partes são ditos \textbf{Autoridades}
  para essas partes. As informações de autoridades são organizadas em
  unidades chamadas \textbf{Zonas}.

\item \textbf{Resolvedores}, são programas que extraem informação dos
  servidores de nomes em resposta a uma requisição de um
  cliente. Resolvedores devem ter acesso a pelo menos um servidor de
  nomes, e usar a informação desse servidor para responder à requisição
  diretamente, ou buscar a resposta usando referências para outros
  servidores de nomes. Um resolvedor é geralmente uma Rotina do Sistema,
  que pode ser diretamente acessada por programas do usuário, sem
  necessidade de um protocolo entre ambos.
\end{enumerate}

Em nome da eficiência, esse três elementos podem ser acoplados em
algumas implementações (por exemplo, um servidor de nomes pode dividir
seu cache com um resolvedor, melhorando o tempo de resposta no caso
geral).

\section{Nomes de Domínio e Registros}

\subsection{Nomes de Domínio}
Como já citado anteriormente, o Sistema de Nomes de Domínios é
organizado em estrutura de árvore, onde cada nodo e cada folha contém um
conjunto de informações, podendo este conjunto ser vazio
\cite{rfc1034}. Cada nodo deve ter um rótulo, cujo tamanho varia entre 0
e 63 octetos, sendo que um nome, o de tamanho zero (nulo), é reservado
para a raiz. Nodos irmãos não podem ter o mesmo nome, mas não há
problema se nodos não irmãos tiverem nomes iguais.

O nome de domínio de um nodo é a lista de rótulos do nodo até a
raiz. Por convenção, os nomes são escritos e lidos da esquerda para a
direita, do mais específico (nome mais perto do nó) para o menos
específico (nodo mais perto da raiz). Os resolvedores tratam da mesma
forma letras maiúsculas e minúsculas, mas os nomes de domínios devem
guardar rótulos sem ignorar essa diferença, mantendo o nome original do
domínio. A justificativa dessa decisão é que, eventualmente, pode ser
necessário diferenciar nomes em caixa alta e caixa baixa, e a
implementação atual garante que nenhum serviço tenha que ser
drásticamente alterado \cite{rfc1034}.

Para simplificar implementações do DNS, o tamanho máximo de octetos de
um nome de domínio é limitado a 255.

\subsection{Registros}

Cada nodo da árvore de nomes de domínio possui um conjunto de
informações de registros, que pode ser vazio. A ordenação de registros
em um conjunto não é significante, e não precisa ser mantida. Assumimos
que um registro tem as seguintes informações \cite{comer}:

\begin{tabular}{ l p{0.82\textwidth} }
\textbf{Dono} & Em qual nome de domínio esse registro é
  encontrado.\\
\textbf{Tipo} & É um código de 16 bits o tipo de recurso
  desse registro. Como padrão, são especificados os seguintes tipos
  \cite{rfc1034}:\\

  & \begin{tabular}{ l p{0.70\textwidth} }
      \emph{A} & Endereço de um servidor\\
      \emph{CNAME} & Nome Canônico de um \textit{Alias}\\
      \emph{HINFO} & Traz informações sobre CPU e Sistema Operacional do
      servidor.\\
      \emph{MX} & Informações sobre o comutador de \textit{e-mails}
      (\textit{Mail eXchanger}) associado ao servidor.\\
      \emph{NS} & Nome do servidor com autoridade sobre o domínio.\\
      \emph{PTR} & Ponteiro para outro domínio de nomes.\\
      \emph{SOA} & \textit{Start of Authority} -- Identifica o início de
      uma Zona de Autoridade.
    \end{tabular}
\end{tabular}
