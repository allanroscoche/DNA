
%%%%%%%%%%%%%%%%%%%%%%%%%%%%%%%%%%%%%%%%%%%%%%%%%%%%%
% Modelo para escrever TCCs, disserta��es e teses utilizando LaTeX, ABNTeX e BibTeX
% Autor/E-Mail: Robinson Alves Lemos/contato@robinson.mat.br/robinson.a.l@bol.com.br
% Data: 19/04/2008 
% Colaboradore(s)/E-Mail(s):
% Caso queira colaborar, entre em contato pelo e-mail e informe altera��es que realizou.
%%%%%%%%%%%%%%%%%%%%%%%%%%%%%%%%%%%%%%%%%%%%%%%%%%%%%
Um pr�-requisito para a compreens�o completa da biologia de um organismo
� determinar toda a sua sequ�ncia gen�tica. Agora nesse momento, h�
v�rios projetos de sequenciamento gen�tico em andamento para v�rios
organismos. Uma parte importante desses projetos � a montagem das partes do
DNA desses seres vivos, onde t�cnicas cada vez mais sofisticadas de computa��o vem
sendo utilizadas para superar esse desafio.

[introduzir os conceitos iniciais na introdu��o como: sequenciador,
  montagem, reads, contigs..]
