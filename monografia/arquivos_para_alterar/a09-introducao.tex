
%%%%%%%%%%%%%%%%%%%%%%%%%%%%%%%%%%%%%%%%%%%%%%%%%%%%%
% Modelo para escrever TCCs, disserta��es e teses utilizando LaTeX, ABNTeX e BibTeX
% Autor/E-Mail: Robinson Alves Lemos/contato@robinson.mat.br/robinson.a.l@bol.com.br
% Data: 19/04/2008 
% Colaboradore(s)/E-Mail(s):
% Caso queira colaborar, entre em contato pelo e-mail e informe altera��es que realizou.
%%%%%%%%%%%%%%%%%%%%%%%%%%%%%%%%%%%%%%%%%%%%%%%%%%%%%
Um pr�-requisito para a compreens�o completa da biologia de um organismo
� determinar toda a sua sequ�ncia gen�tica.
Agora nesse momento, h� v�rios projetos de sequenciamento gen�tico em andamento para v�rios
organismos. 
Novas tecnologias de sequenciamento s�o capazes de gerar bilh�es de bases gen�ticas \cite{Harismendy2009}.
Lidar com esses dados criam cada vez mais desafios computacionais.
O grande n�mero de informa��es em conjunto com sua complexidade faz com que mais recursos computacionais sejam utilizados e algoritmos mais sofisticados implementados.


No Cap�tulo 1 � apresentado o sequenciamento gen�tico, sequenciadores de nova gera��o(SNG, ingles: NGS). Ser� colocado em foco o sequenciador \emph{ABI SOLiD4} da \emph{Life Tecnologies} \cite{Life}, juntamente com seu tipo de dados exclusivo gerado por sua tecnologia de sequenciamento.
O Cap�tulo 2 mostra o pipeline de programas utilizado para montagens de novos genomas. Cada programa ser� abordado brevemente com exce��o do velvet.
Por fim o Cap�tulo 3 fala de alguns problemas com m� utiliza��o de recursos computacionais juntamente com alguns desafios envolvidos com o tratamento de erros.
%Uma parte importante desses projetos � a montagem das partes do DNA desses seres vivos, onde t�cnicas cada vez mais sofisticadas de computa��o vem sendo utilizadas para superar esse desafio.


%[introduzir os conceitos iniciais na introdu��o como: sequenciador,
%  montagem, reads, contigs..]
