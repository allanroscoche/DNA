Em meados dos anos 1980, com o crescimento desenfreado da rede ARPANet, foi necess�rio desenvolver um sistema escal�vel de tradu��o de nomes para endere�os v�lidos. Os requisitos exigiam um sistema robusto, com possibilidade de expans�o e que suportasse extens�es. Nascia a� o protocolo DNS (\textit{Domain Name System}). Em 2008 foi encontrada uma das maiores falhas de seguran�a da hist�ria do protocolo DNS. Sua corre��o foi amplamente divulgada, mas pouco se questionou sobre a efetividade das solu��es adotadas, em sua maioria inspiradas nas solu��es usadas na implementa��o \textit{DJBDNS}. O objetivo desse trabalho � analisar a natureza da falha encontrada em 2008, os riscos que ela propiciava, a efetividade da corre��o sugerida tanto pr�tica quanto te�rica -- uma vez que a descri��o do protocolo teve que ser alterada.

Palavras-chave: DNS, falha de seguran�a, \textit{DJBDNS}, protocolo, ARPANET

%%%%%%%%%%%%%%%%%%%%%%%%%%%%%%%%%%%%%%%%%%%%%%%%%%%%%
% Modelo para escrever TCCs, disserta��es e teses utilizando LaTeX, ABNTeX e BibTeX
% Autor/E-Mail: Robinson Alves Lemos/contato@robinson.mat.br/robinson.a.l@bol.com.br
% Data: 19/04/2008 
% Colaboradore(s)/E-Mail(s):
% Caso queira colaborar, entre em contato pelo e-mail e informe altera��es que realizou.
%%%%%%%%%%%%%%%%%%%%%%%%%%%%%%%%%%%%%%%%%%%%%%%%%%%%%