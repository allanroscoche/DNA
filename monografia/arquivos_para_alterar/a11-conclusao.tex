Os sequenciadores est�o se aperfei�oando cada vez mais e produzindo
quantidades cada vez maiores de informa��o, com isso os desafios
computacionais de montadores tem ser tornado cada vez maiores. Uma das
maiores dificuldades em compreender os algoritmos e implementa��o dos
montadores � a falta de informa��o b�sica sobre o assunto, j� que v�rios
conceitos de biologia e de computa��o s�o necess�rios para o compreens�o
do seu funcionamento. O trabalho tenta atrav�s de uma implementa��o mais
simples e econ�mica de um montador a possibilidade de realizar a
montagem de c�digo gen�tico num computador de mesa e poder compreender
quais as mudan�as no montador afetam a velocidade e a qualidade. Muitos
dos testes realizados utilizando dados reais acabavam com a mem�ria ou
tinham o tempo de execu��o impratic�veis para um computador de
mesa. Atrav�s dessa implementa��o uma das partes de um montador pode ser
realizada formando alguns ``grudes'' na estrat�gia do grafo de bruijn.

%%%%%%%%%%%%%%%%%%%%%%%%%%%%%%%%%%%%%%%%%%%%%%%%%%%%%
% Modelo para escrever TCCs, disserta��es e teses utilizando LaTeX, ABNTeX e BibTeX
% Autor/E-Mail: Robinson Alves Lemos/contato@robinson.mat.br/robinson.a.l@bol.com.br
% Data: 19/04/2008 
% Colaboradore(s)/E-Mail(s):
% Caso queira colaborar, entre em contato pelo e-mail e informe altera��es que realizou.
%%%%%%%%%%%%%%%%%%%%%%%%%%%%%%%%%%%%%%%%%%%%%%%%%%%%%
