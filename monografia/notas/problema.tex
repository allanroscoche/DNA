O DNA é um composto orgânico cujas moléculas contêm as instruções genéticas
 que coordenam o desenvolvimento e funcionamento de todos os seres vivos e alguns vírus.
Ele é composto por uma longa cadeia de nucleotídios que se diferenciam pela sua base hidrogenada
que vamos chamar de A (adenina), C (citosia), T(timina) e G(guanina). O objetivo do sequenciamento
genético é descobrir a sequência de bases hidrogenadas que formam o código genético de um organismo.

O processo do sequenciamento consegue descobrir um número limitado de bases por vez. O conjunto de bases
contínuas é chamado de leitura, e através de várias leituras é possível descobrir todo o código genético
do organismo. A representação dessas leituras pode ser feita através de strings em que cada letra representa uma base.
A [fig. 1] mostra um exemplo de um arquivo de leituras.

Para se obter o código genético completo é necessário comparar e organizar as leituras, para isso são utilizados
montadores. Os montadores são responsáveis por organizar as leituras para corresponder ao que está no genoma.




- Repetições

- Por que o velvet não monta o dna completo

- Sobreposição

- Os montadores geram os contigs, as partes do genoma não coberto pelos contigs são chamados de gaps, os gaps são sequenciados usando o sequenciamento dirigido.



