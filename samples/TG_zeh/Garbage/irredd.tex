%%%%%%%%%%%%%%%%%%%%%%%%%%%%%%%%%%%%%%%%%%%%%%%%%%%%%%%%%%%%%%%%%%%%%%%%
% irredd.tex                                                           %
%                                                    início do arquivo %
%%%%%%%%%%%%%%%%%%%%%%%%%%%%%%%%%%%%%%%%%%%%%%%%%%%%%%%%%%%%%%%%%%%%%%%%

\chapter{Irredutibilidade}

\section{Um teste de irredutibilidade}

\begin{Teo}\label{maint}
  Sendo $n$ um natural não nulo, $p$ um primo positivo,
  \begin{equation*}
    L=\cjpp{\ell_j}{1\leq j\leq n}
  \end{equation*}
  o
  conjunto de todos os divisores primos de $n$, $k=|L|$ e
  \begin{equation*}
    m_j=\frac{n}{\ell_j}\MMv\quad\forall j\in[k]\MMv
  \end{equation*}
  um polinômio $g\in\MMZ_p[x]$ de grau
  $n$ é irredutível em $\MMZ_p[x]$ se e só se:
  \begin{enumerate}[({\ref{maint}}.i)]
    \item\label{mainti} $g(x)\divd \bigl(x^{p^n} -x\bigr)$;
    \item\label{maintii} para todo $j\in[k]$, o polinômio constante
      $1(x)$
    é o único polinômio que divide ambos $g(x)$ e $x^{p^{m_j}}-x$
  \end{enumerate}
\end{Teo}

\begin{prova}
  Assumamos inicialmente que $g(x)$ seja irredutível em $\MMZ_p[x]$.
  Do
  teorema\xspace\ref{everyroot}, temos que toda raiz $\alpha$ de $g(x)$
  pertence a
  $\galois{p^n}$. Como $\galois{p^n}$ é um corpo finito com $p^n$
  elementos, temos, do teorema\xspace\ref{teoacardifa}, que
  \begin{equation*}
    \alpha^{p^n} = \alpha\MMv
  \end{equation*}
  o que caracteriza $\alpha$ como raiz do polinômio $x^{p^n}-x$ e
  nos traz, do teorema\xspace\ref{teoraizdivide}, que
  \begin{equation*}
    (x-\alpha)\divd \bigl(x^{p^n}-x\bigr)\MMp
  \end{equation*}
  Logo,
  \begin{equation*}
    g(x)\divd \bigl(x^{p^n} -x\bigr)
  \end{equation*}
  e provamos (\ref{mainti}). Ademais, temos que,
  para qualquer natural $m$
  menor que $n$, $g(x)$ não tem raízes em $\galois{p^m}$. Portanto, para
  toda raiz $\alpha$ de $g(x)$ e todo natural $m$ menor que $n$,
  sabemos,
  também do teorema\xspace\ref{teoraizdivide},
  que
  \begin{equation*}
    (x-\alpha)\ndivd \bigl(x^{p^m}-x\bigr)\MMp
  \end{equation*}
  Como para todo divisor $d(x)$ de $g(x)$ diferente do polinômio
  constante
  $\um(x)$
  existe um subconjunto $\Gamma$ do conjunto das raízes de $g(x)$
  tal que
  \begin{equation*}
    \prod_{\gamma\in\Gamma}(x-\gamma) = d(x)\MMv
  \end{equation*}
  temos que nenhum divisor de $g(x)$ diferente do polinômio constante
  $\um(x)$
  divide $x^{p^m}-x$, sendo $m$ um natural menor que $n$. Assim, em
  particular, temos que nenhum divisor de $g(x)$ diferente do polinômio
  constante $\um(x)$ divide $x^{p^m_j}-x$, para todo $j\in[k]$, e
  provamos (\ref{maintii}).

  Inversamente, assumamos agora (\ref{mainti}) e (\ref{maintii}). Como
  \begin{equation*}
    (x-\alpha)\divd \bigl(x^{p^n}-x\bigr)\MMv
  \end{equation*}
  qualquer que seja $\alpha$ raiz de $g(x)$, todas as raízes de $g(x)$
  estão em $\galois{p^n}$. Queremos demonstrar que $g$ é
  irredutível. Suponhamos, entretanto, que $g$ seja redutível e tomemos
  $g_1$ um divisor irredutível não constante
  de $g$. Sendo $m$ o grau de $g_1$, com
  $m<n$, sabemos, do teorema\xspace\ref{everyroot},
  que todas as raízes de $g_1(x)$ pertencem a $\galois{p^m}$, que é
  gerado sobre $\MMZ_p$ por qualquer uma dessas raízes,
  conforme o mesmo teorema\xspace\ref{everyroot}.
  Já que $g_1$ é um divisor de $g$, temos que $\galois{p^m}\subseteq
  \galois{p^n}$ e que $m\divd n$. Como $m<n$, é trivial que $m\divd
  m_t$ para algum $t\in [k]$. Assim, analogamente, todas as
  raízes de $g_1$ estão em $\galois{p^{m_t}}$. Portanto, $g_1(x)$,
  que não é o polinômio constante $\um(x)$,
  divide ambos $g(x)$ e $x^{p^{m_t}}-x$,
  o que contraria
  (\ref{maintii}). Conseqüentemente, $g(x)$ é irredutível.
\end{prova}



%%%%%%%%%%%%%%%%%%%%%%%%%%%%%%%%%%%%%%%%%%%%%%%%%%%%%%%%%%%%%%%%%%%%%%%%
% irred.tex                                                            %
%                                                       fim do arquivo %
%%%%%%%%%%%%%%%%%%%%%%%%%%%%%%%%%%%%%%%%%%%%%%%%%%%%%%%%%%%%%%%%%%%%%%%%
