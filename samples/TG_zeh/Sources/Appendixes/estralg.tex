%%%%%%%%%%%%%%%%%%%%%%%%%%%%%%%%%%%%%%%%%%%%%%%%%%%%%%%%%%%%%%%%%%%%%%%%
% estralg.tex                                                          %
%                                                    início do arquivo %
%%%%%%%%%%%%%%%%%%%%%%%%%%%%%%%%%%%%%%%%%%%%%%%%%%%%%%%%%%%%%%%%%%%%%%%%

\chapter{Fundamentos algébricos}

\section{Conceitos preliminares}

\subsection{Partições}

\begin{Def}
  Sendo $A$ um conjunto qualquer, o \conceito{conjunto potência}, ou
  \conceito{conjunto das partes}, de $A$, denotado por
  $\simb[conjunto potência (ou conjunto das partes) de $A$]{2^A}$,
  é o conjunto de todos os subconjuntos de $A$.
\end{Def}

\begin{Def}
  Sendo $A$ um conjunto e $\mathscr{F}$ um subconjunto de $2^A$, dizemos
  que $\mathscr{F}$ é uma \conceito{partição} de $A$ se e só se:
  \begin{enumerate}[(i)]
    \item nenhum elemento de $\mathscr{F}$ é vazio;
    \item dois elementos de $\mathscr{F}$  ou são iguais ou são
    disjuntos;
    \item a união de todos os elementos de $\mathscr{F}$ é o próprio
    conjunto $A$.
  \end{enumerate}
\end{Def}

\subsection{Relações}

\begin{Def}
  Sendo $A$ e $B$ conjuntos quaisquer, uma \conceito{relação} $\sim$
  entre
  $A$ e $B$ é qualquer subconjunto de $A\times B$. Mais especificamente,
  se $A=B$, diz-se que $\sim$ é uma relação sobre $A$, ou uma relação em
  $A$.
\end{Def}

\begin{Not}
  Sendo $\sim$ uma relação entre um conjunto $A$ e um conjunto $B$,
  escrevemos, para todo $a\in A$ e todo $b\in B$,
  $\simb[relação $\sim$ entre $a$ e $b$]{a\sim b}$
  sempre que
  $(a,b)\in\sim$ e
  $\simb[negação da relação $\sim$ entre $a$ e $b$]{a\nsim b}$
  sempre que $(a,b)\notin\sim$.
\end{Not}

\begin{Nom}
  Dizemos que uma relação $\sim$ sobre um conjunto $A$
  é \Conceito{reflexiva}{relação reflexiva} quando e só quando
  se $a\sim a$ para
  todo elemento $a$ e $A$.
\end{Nom}

\begin{Nom}
  Dizemos que uma relação $\sim$ sobre um conjunto $A$
  é \Conceito{simétrica}{relação simétrica} quando e só quando,
  para todo $a$ e todo $b$ elementos de $A$, vale que se
  $a\sim b$ então $b\sim a$.
\end{Nom}

\begin{Nom}
  Dizemos que uma relação $\sim$ sobre um conjunto $A$
  é \Conceito{transitiva}{relação transitiva} quando e só quando,
  para quaisquer $a$, $b$ e $c$ elementos de $A$, vale que se
  $a\sim b$ e $b\sim c$ então $a\sim c$.
\end{Nom}

\begin{Nom}
  Dizemos que uma relação $\sim$ sobre um conjunto $A$
  é \Conceito{an\-ti\--si\-mé\-tri\-ca}{relação
  anti-simétrica}
  quando e só quando, para todo $a$ e todo
  $b$ elementos de $A$, vale que se
  $a\sim b$ e $b\sim a$ então $a=b$.
\end{Nom}

\begin{Nom}
  Dizemos que uma relação $\sim$ sobre um conjunto não vazio $A$ é
  uma \conceito{relação de equivalência} sobre $A$ se e só se $\sim$ é
  reflexiva, simétrica e transitiva.
\end{Nom}

\begin{Def}\label{defcleq}
  Sendo $\sim$ uma relação de equivalência sobre um conjunto não vazio
  $A$ e $a$
  um elemento de $A$, a
  \conceito{classe de equivalência} de $a$ por $\sim$ é o
  conjunto
  \begin{equation*}
    \simb[classe de equivalência de $a$ por $\sim$]{\cleq{a}{\sim}}
    = \cjpp{b\in A}{b\sim a}
  \end{equation*}
\end{Def}

\begin{Def}
  O \conceito{conjunto quociente} de um conjunto não vazio
  $A$ por uma relação
  $\sim$ sobre $A$, denotado por
  $\simb[conjunto quociente de $A$ por $\sim$]{\tcquoc{A}{\sim}}$
  é o conjunto
  \begin{equation*}
    \cquoc{A}{\sim} =
    \cjpp{\cleq{a}{\sim}}{a\in A}
  \end{equation*}
\end{Def}

\begin{Teo}\label{teocquoc}
  O conjunto quociente de um conjunto não vazio
  $A$ por uma relação
  $\sim$ sobre $A$
  é uma partição de $A$.
\end{Teo}

\begin{dem}
  Como
  $A$ não é vazio, seja $y$ qualquer um de seus elementos. É imediato
  que $\cleq{y}{\sim}$ não seja vazio, já que pelo menos
  $y\in\cleq{y}{\sim}$,
  uma vez que $\sim$
  se trata de uma relação reflexiva.

  Agora, sejam $a$ e $b$ elementos de $A$. Sabemos que se
  $\cleq{a}{\sim}=\cleq{b}{\sim}$
  então
  $\cleq{a}{\sim}$ e $\cleq{b}{\sim}$ não são disjuntos, pois nem
  $\cleq{a}{\sim}$
  nem $\cleq{b}{\sim}$ é
  vazio. Para mostrarmos que se $\cleq{a}{\sim}\neq \cleq{b}{\sim}$
  então
  $\cleq{a}{\sim}$ e $\cleq{b}{\sim}$ são
  disjuntos, utilizaremos a forma contrapositiva. Suponhamos que
  $\cleq{a}{\sim}$ e
  $\cleq{b}{\sim}$ não sejam conjuntos disjuntos e tomemos $x$ um
  elemento
  qualquer
  de $\cleq{a}{\sim}\cap\cleq{b}{\sim}$. Assim, $x\sim a$; portanto,
  $a\sim
  x$, e, como $x\sim
  b$, $a\sim b$ e $b\sim a$.
  Logo, vale que, para todo $\alpha\in\cleq{a}{\sim}$, $\alpha\sim
  b$, já que $\alpha\sim a$ e $a\sim b$, e, conseqüentemente, que
  $\alpha\in\cleq{b}{\sim}$, o que nos leva a concluir que
  $\cleq{a}{\sim}\subseteq \cleq{b}{\sim}$. Por
  outro lado, vale também que, para todo $\beta\in\cleq{b}{\sim}$,
  $\beta\sim
  a$,
  já que $\beta\sim b$ e $b\sim a$, e, conseqüentemente, que
  $\beta\in\cleq{a}{\sim}$, o que nos leva a concluir que
  $\cleq{b}{\sim}\subseteq \cleq{a}{\sim}$. Por
  fim, mostramos o que queríamos: que $\cleq{a}{\sim}=\cleq{b}{\sim}$.

  Resta-nos ainda concluir que
  \begin{equation*}
    \bigcup_{a\in A}\cleq{a}{\sim} = A\MMp
  \end{equation*}
  Da definição de classe de equivalência segue naturalmente que
  $\cleq{a}{\sim}\subseteq A$ para todo $a\in A$, e, portanto,
  \begin{equation*}
    \bigcup_{a\in A}\cleq{a}{\sim} \subseteq A\MMp
  \end{equation*}
  Tomemos agora
  $x\in A$. Como $\sim$ é reflexiva, é imediato que $x\in[x]$, e,
  dessarte, que $x\in\bigcup_{a\in A}\cleq{a}{\sim}$. Finalmente,
  demonstramos que
  \begin{equation*}
    A \subseteq \bigcup_{a\in A}\cleq{a}{\sim}
  \end{equation*}
  e encerramos nossa prova.
\end{dem}

\subsection{Divisibilidade no conjunto dos números inteiros}

O conceito de divisibilidade no conjunto dos números inteiros está no
núcleo da Teoria dos Números. A partir dele serão desenvolvidos os
principais e mais interessantes conceitos e resultados da teoria que
hoje possui aplicações inclusive para Criptografia e segurança de
sistemas.

\begin{Def}\label{defdivide}
  Dizemos que um número inteiro $d$ divide um número inteiro $z$, e
  escrevemos $\simb[divisibilidade do inteiro $d$ pelo inteiro
    $z$]{d\divd z}$, se e só
  se existe um número inteiro $k$ tal que $z=dk$.
\end{Def}

\begin{Nom}\label{nomdivinteiro}
  Quando $d\divd z$, sendo $d$ e $z$ inteiros, dizemos que:
  \begin{enumerate}[(i)]
    \item $z$ é \Conceito{divisível}{divisibilidade} por $d$;
    \item $z$ é \Conceito{múltiplo}{múltiplo de um número inteiro} de
    $d$;
    \item $d$ é \Conceito{divisor}{divisor de um número inteiro} de $z$;
    \item $k$ é o \Conceito{quociente da divisão}{quociente de uma
      divisão} de $z$ por $d$, sendo $k$ um inteiro tal que $z=dk$.
  \end{enumerate}
\end{Nom}

\begin{Propr}\label{dividezero}
  Qualquer número inteiro divide $0$.
\end{Propr}

\begin{dem}
  Sendo $d$ um inteiro qualquer, é verdade que $0=d\cdot 0$. Portanto,
  $d\divd 0$, tendo $0$ como quociente.
\end{dem}

\begin{Propr}\label{zeronaodivide}
  $0$ divide somente o próprio $0$.
\end{Propr}

\begin{dem}
  É imediato que $0\divd 0$, uma vez que $0=0k$ não importando o inteiro
  $k$. Temos então apenas de mostrar que se $0\divd z$ então $z=0$,
  qualquer que seja o inteiro $z$. Ora, se $0\divd z$ então existe um
  inteiro $k$ tal que $z = 0k$ e, portanto, $z=0$, como queríamos
  mostrar.
\end{dem}

\begin{Propr}\label{modleq}
  Sendo $d$ e $z$ inteiros, se $d\divd z$ então $|d|\leq|z|$.
\end{Propr}

\begin{dem}
  Suponhamos que $d\divd z$ e tomemos um inteiro $k$ tal
  que $z = dk$. Assim, verifica-se que $|z| = |d||k|$ e,
  conseqüentemente, que $|z|\geq |d|$.
\end{dem}

\begin{Propr}
  Apenas os inteiros $1$ e $-1$ dividem o número $1$.
\end{Propr}

\begin{dem}
  Seja $x$ um inteiro que divida $1$. Da propriedade\xspace\ref{modleq},
  $|x|\leq 1$; portanto, $x\in\cj{0,1,-1}$. Como,
  da propriedade\xspace\ref{zeronaodivide}, $0\ndivd 1$, e como
  $1\divd 1$
  e $-1\divd 1$, temos que apenas $1$ e $-1$
  dividem $1$.
\end{dem}

\begin{Propr}
  $z\divd z$ para todo $z\in\MMZ$.
\end{Propr}

\begin{dem}
  Seja $z$ um inteiro. Como $z = z\cdot 1$, é verdade que $z\divd z$.
\end{dem}

\begin{Propr}\label{divideoposto}
  Sendo $d$ e $z$ inteiros, se $d\divd z$ então $d\divd -z$.
\end{Propr}

\begin{dem}
  Se $d\divd z$ então existe um $k$ inteiro
  tal que $z = dk$ e, portanto, $-z = -dk = d(-k)$. Como $-k$ é inteiro,
  $d\divd -z$.
\end{dem}

\begin{Propr}
  Sendo $a$, $b$ e $c$ inteiros, se $a\divd b$ e $b\divd c$ então
  $a\divd c$.
\end{Propr}

\begin{dem}
  Suponhamos que $a\divd b$ e que
  $b\divd c$. Assim, existem $k_1$ e $k_2$ inteiros tais que $b=ak_1$ e
  $c=bk_2$.
  Portanto,
  \begin{equation*}
      c = bk_2
        = (ak_1)k_2
        = a(k_1k_2)\MMv
  \end{equation*}
  e, como $(k_1k_2)$ é um número inteiro, $a\divd c$.
\end{dem}

\begin{Propr}
  Sendo $a$, $b$, $c$ e $d$ inteiros,
  se $a\divd b$ e $c\divd d$ então
  $ac\divd bd$.
\end{Propr}

\begin{dem}
  Suponhamos que $a\divd b$ e que
  $c\divd d$. Assim, existem $k_1$ e $k_2$ inteiros tais que $b=ak_1$ e
  $d=ck_2$.
  Portanto,
  \begin{equation*}
      bd = (ak_1)(ck_2)
         = (ac)(k_1k_2)\MMv
  \end{equation*}
  e, como $(k_1k_2)$ é um número inteiro, $ac\divd bd$.
\end{dem}

\begin{Propr}
  Se um inteiro $d$ divide um inteiro $z$ então $d$ divide qualquer
  múltiplo de $z$.
\end{Propr}

\begin{dem}
  Sejam $d$ e $z$ inteiros tais que $d\divd z$. Seja $m$ um inteiro
  qualquer. Queremos mostrar que $d\divd mz$. Sendo $k$ um inteiro tal
  que $z = dk$, temos que $mz = m(dk)$ e, assim, que $mz = d(mk)$. Como
  $mk$ é um inteiro, concluímos que $d\divd mz$.
\end{dem}

\begin{Propr}\label{dividecombinacaolinear}
  Sendo $a$, $b$ e $c$ inteiros, se $a\divd b$ e $a\divd c$ então
  \begin{equation*}
    a\divd (mb+nc)\MMv
  \end{equation*}
  quaisquer que sejam $m$ e $n$ números inteiros.
\end{Propr}

\begin{dem}
  Suponhamos que $a\divd b$ e que
  $a\divd c$. Assim, existem $k_1$ e $k_2$ inteiros tais que $b = ak_1$
  e $c = ak_2$.
  Portanto, sendo $m$ e $n$ inteiros quaisquer,
  \begin{equation*}
    \begin{aligned}
      mb+nc &= m(ak_1) + n(ak_2)\\
            &= a(mk_1) + a(nk_2)\\
            &= a(mk_1+nk_2)\MMp
    \end{aligned}
  \end{equation*}
  Assim, já que $(mk_1+nk_2)$ é um número inteiro, concluímos que
  $a\divd (mb+nc)$.
\end{dem}

\begin{Propr}
  Um inteiro $d$ divide um inteiro $z$ se e somente se $d$ divide
  $|z|$.
\end{Propr}

\begin{dem}
  Se $d\divd z$ então existe um inteiro $k$
  tal que $z = dk$ e, portanto, como
  \begin{equation*}
    |z| = |dk| = |d||k| = sd|k| = d(s|k|)\MMv
  \end{equation*}
  $d\divd |z|$, já que $(s|k|)$ é inteiro,
  sendo
  \begin{equation*}
    s = \left\{
    \begin{aligned}
      &1\MMv&\quad&\text{se $d\geq 0$;}\\
      &-1\MMv&\quad&\text{caso contrário.}
    \end{aligned}
    \right.
  \end{equation*}
  Por outro lado, se $d\divd|z|$ então existe um inteiro $k$ tal que
  $|z| = dk$ e, portanto, como $z = s|z|$, temos que
  \begin{equation*}
    z = s|z| = sdk = d(sk)\MMv
  \end{equation*}
  sendo
  \begin{equation*}
    s = \left\{
    \begin{aligned}
      &1\MMv&\quad&\text{se $z\geq 0$,}\\
      &-1\MMv&\quad&\text{caso contrário.}
    \end{aligned}
    \right.
  \end{equation*}
  Por fim, como $(sk)$ é um inteiro,
  temos que $d\divd z$ e concluímos a demonstração.
\end{dem}

\begin{Propr}\label{primodivide}
  Se um número primo $p$ divide $ab$, sendo $a$ e $b$ inteiros, então
  $p\divd a$ ou $p\divd b$.
\end{Propr}

\begin{dem}
  Suponhamos que
  $p\divd ab$ mas que $p\ndivd a$ e que $p\ndivd b$. Assim, $p$ não é
  um dos primos da fatoração de $a$ nem tampouco algum dos primos da
  fatoração de $b$, o que contraria o teorema fundamental da Aritmética,
  já que a fatoração de $ab$ é única e composta pelos primos de $a$ e de
  $b$.
\end{dem}

A seguir, apresentamos a definição da principal relação de equivalência
em Teoria dos Números.

\begin{Def}
  Dizemos que um inteiro $a$ é \Conceito{côngruo}{congruência} a um
  inteiro $b$ módulo um inteiro $m$, e escrevemos
  \begin{equation*}
    \simb[congruência entre $a$ e $b$ módulo $m$]{\congmod{a}{b}{m}}
  \end{equation*}
  se e só se $m\divd (a-b)$.
\end{Def}

\begin{Teo}
  Sendo $m$ um inteiro não nulo,
  a relação de congruência é uma relação de
  equivalência sobre $\MMZ$.
\end{Teo}

\begin{dem}
  É imediato que, para todo $z\in\MMZ$,
  \begin{equation*}
    \congmod{z}{z}{m}\MMv
  \end{equation*}
  já que $m\divd(z-z)$, conforme a
  propriedade\xspace\ref{dividezero}.

  Sejam $a$ e $b$ inteiros. Se
  \begin{equation*}
    \congmod{a}{b}{m}
  \end{equation*}
  então $m\divd (a-b)$ e, portanto, da
  propriedade\xspace\ref{divideoposto}, $m\divd(b-a)$, e,
  conseqüentemente,
  \begin{equation*}
    \congmod{b}{a}{m}\MMp
  \end{equation*}

  Resta-nos ainda mostrar a transitividade da congruência. Para
  tanto, tomemos $a$, $b$ e $c$ inteiros tais que
  \begin{equation*}
    \begin{aligned}
      \congmoda{a}{b}{m}\quad\text{e}\\
      \congmoda{b}{c}{m}\MMp
    \end{aligned}
  \end{equation*}
  Como $m\divd (a-b)$ e $m\divd (b-c)$, temos, da
  propriedade\xspace\ref{dividecombinacaolinear}, que
  \begin{equation*}
    m\divd\bigl((a-b)+(b-c)\bigr)
  \end{equation*}
  e, dessarte, que
  \begin{equation*}
    m\divd(a-c)
  \end{equation*}
  e, assim, que
  \begin{equation*}
    \congmod{a}{c}{m}\MMv
  \end{equation*}
  como queríamos mostrar.
\end{dem}

\begin{Not}\label{notcleqint}
  Sendo $m$ um inteiro não nulo,
  escrevemos $\MMZ_m$ para denotar o conjunto
  quociente de $\MMZ$ pela relação de congruência módulo $m$. Além
  disso, sendo $z$ um inteiro,
  utilizamos $\simb[classe de equivalência de $z$ pela relação de
  congruência módulo $m$]{\cleq{z}{m}}$ para denotar a classe de
  equivalência de $z$
  por essa mesma relação.
\end{Not}

\subsection{Funções}

Um caso particular de relações são as chamadas ``funções'', um dos
conceitos mais antigos --- ao menos intuitivamente --- e de maiores
aplicações na Engenharia e Tecnologia. É do estudo das funções que
surgiu o Cálculo Diferencial e Integral, considerada uma das disciplinas
mais importantes da Matemática Contínua Aplicada.

\begin{Def}
  Sendo $f$ uma relação entre um conjunto qualquer $A$ e um conjunto
  qualquer $B$, dizemos que $f$ é uma \conceito{função}, ou
  \conceito{aplicação},
  de $A$ em $B$ e escrevemos $\simb[função de $A$ em
  $B$]{\funcao{f}{A}{B}}$ se e somente se
  para todo elemento $a$ de $A$ existir um e apenas um elemento $b$ de
  $B$ tal que $afb$.
\end{Def}

\begin{Nom}
  Sendo $f$ uma função de um conjunto $A$ num conjunto $B$,
  dizemos que $A$ é o
  \Conceito{domínio}{domínio de uma função} de $f$ e que $B$, o
  \Conceito{contradomínio}{contradomínio de uma função} de $f$.
\end{Nom}

\begin{Nom}
  Sendo $f$ uma função de um conjunto $A$ num conjunto $B$
  e sendo $a$ um elemento de $A$,
  a \conceito{imagem} de $a$ por $f$, denotado por
  $\simb[imagem de $a$ por $f$]{f(a)}$,
  é o único elemento
  $b$ de $B$ tal que $afb$.
\end{Nom}

\begin{Not}
  Sendo $f$ uma função de um conjunto $A$ num conjunto $B$
  e sendo $S$ um subconjunto de $A$,
  usamos $\simb[conjunto das imagens dos elementos de $S$ por
  $f$]{f(S)}$ para denotar o conjunto
  \begin{equation*}
    f(S) = \cjpp{f(s)}{s\in S}\MMp
  \end{equation*}
\end{Not}

\begin{Def}
  Sendo $f$ uma função de um conjunto $A$ num conjunto não vazio $B$
  e sendo $b$ um elemento de $B$,
  a \conceito{imagem inversa} de $b$ por $f$, denotada por
  $\simb[imagem inversa de $b$ por $f$]{f^{-1}(b)}$,
  é o conjunto
  \begin{equation*}
    f^{-1}(b) = \cjpp{a\in A}{f(a)=b}\MMp
  \end{equation*}
\end{Def}

\begin{Not}
  Sendo $f$ uma função de um conjunto $A$ num conjunto $B$
  e sendo $S$ um subconjunto de $B$,
  usamos $\simb[conjunto das imagens inversas dos elementos de $S$ por
  $f$]{f^{-1}(S)}$ para denotar o conjunto
  \begin{equation*}
    f^{-1}(S) = \cjpp{f^{-1}(s)}{s\in S}\MMp
  \end{equation*}
\end{Not}

\begin{Not}
  Sendo $A$ e $B$ conjuntos, utilizamos $\simb[conjunto das funções de
  $B$ em $A$]{A^B}$ para denotar o conjunto de
  todas as funções de $B$ em $A$.
\end{Not}

\begin{Nom}
  Dizemos que uma função $f$ de um conjunto $A$ num conjunto $B$ é
  \Conceito{injetiva}{função injetiva}, ou \Conceito{injetora}{função
  injetora}, ou ainda uma \conceito{injeção}, quando e só quando
  vale que se $x$ e
  $y$ são elementos distintos de $A$ então $f(x)\neq f(y)$.
\end{Nom}

\begin{Nom}
  Dizemos que uma função $f$ de um conjunto $A$ num conjunto $B$ é
  \Conceito{sobrejetiva}{função sobrejetiva},
  ou \Conceito{sobrejetora}{função
  sojetora}, ou ainda uma \conceito{sobrejeção}, quando e só quando
  $f(A)=B$.
\end{Nom}

\begin{Nom}
  Dizemos que uma função $f$ de um conjunto $A$ num conjunto $B$ é
  \Conceito{bijetiva}{função bijetiva}, ou \Conceito{bijetora}{função
  bijetora}, ou ainda uma \conceito{bijeção}, quando e só quando $f$ for
  ao mesmo tempo injetiva e sobrejetiva.
\end{Nom}

\begin{Nom}
  Dizemos que dois conjuntos $A$ e $B$ \Conceito{correspondem-se
  bi\-u\-ni\-vo\-ca\-men\-te}{correspondência biunívoca entre dois conjuntos},
  e escrevemos $\simb[correspondência biunívoca entre $A$ e $B$]{A\simeq
  B}$, se e
  só se existe uma bijeção de $A$ em $B$.
\end{Nom}

\subsection{Operações}

Um caso particular de funções,
as operações sobre conjuntos, embora também se tratem de conceitos
bastante antigos intuitivamente, constituem um dos pilares das
estruturas algébricas da Matemática Moderna, como poderemos observar
adiante.

\begin{Def}
  Sendo $A$ um conjunto qualquer, uma \conceito{operação} sobre $A$ é
  qualquer função $\ast$ cujo domínio seja $A\times A$
  e o contradomínio, $A$.
\end{Def}

\begin{Nom}
  Sendo $\ast$ uma operação sobre um conjunto $A$,
  dizemos que $A$ é \Conceito{munido}{conjunto munido de uma
  operação} da operação $\ast$.
\end{Nom}

\begin{Not}
  Sendo $A$ um conjunto munido de uma operação $\ast$ e sendo $a$ e $b$
  elementos de $A$, denotamos $\ast(a,b)$ por
  $\simb[imagem de $(a,b)$ por $\ast$]{a\ast b}$.
\end{Not}

\begin{Nom}
  Dizemos que uma operação $\ast$ sobre um conjunto $A$ é
  \Conceito{associativa}{associatividade}  se e só se, para quaisquer
  $a$, $b$ e $c$ elementos de $A$,
  \begin{equation*}
    a\ast(b\ast c) = (a\ast b)\ast c\MMp
  \end{equation*}
\end{Nom}

\begin{Nom}
  Dizemos que uma operação $\ast$ sobre um conjunto $A$ é
  \Conceito{comutativa}{comutatividade}  se e só se, para quaisquer $a$
  e $b$ elementos de $A$,
  \begin{equation*}
    a\ast b = b\ast a\MMp
  \end{equation*}
\end{Nom}

\begin{Nom}
  Dizemos que uma operação $\triangle$ sobre um conjunto $A$ é
  \Conceito{distributiva}{distributividade} em relação a uma outra
  operação $\ast$ sobre $A$ se e só se, para quaisquer $a$, $b$ e $c$
  elementos de $A$,
  \begin{equation*}
    \begin{aligned}
      a\triangle(b\ast c) &=
      (a\triangle b) \ast (a\triangle c)\qquad\text{e}\\
      (a\ast b)\triangle c &=
      (a\triangle c)\ast(b\triangle c)\MMp
    \end{aligned}
  \end{equation*}
\end{Nom}

\begin{Nom}
  Sendo $A$ um conjunto munido de uma operação $\ast$ e $S$ um
  subconjunto de $A$, dizemos que $S$ é \Conceito{fechado para a
  operação}{conjunto fechado para uma operação} $\ast$ se e só se $r\ast
  s$ pertence a $S$ para quaisquer elementos $r$ e $s$ de $S$.
\end{Nom}

\begin{Def}
  Sendo $A$ um conjunto munido de uma operação $\ast$ e $S$ um
  subconjunto de $A$ fechado para $\ast$, a \Conceito{restrição da
  operação}{restrição de uma operação} $\ast$ a $S$ é a operação
  $\funcao{\simb[restrição de $\ast$ a $S$]{\ast_S}}{S\times S}{S}$
  definida por:
  \begin{equation*}
    r\ast_Ss = r\ast s\MMp
  \end{equation*}
\end{Def}

\begin{Nom}
  Dizemos que um elemento $e$ é um \conceito{elemento neutro}
  para uma operação
  $\ast$ sobre um conjunto $A$ se e só se vale que
  \begin{equation*}
    a\ast e = e\ast a = a\MMv
  \end{equation*}
  para todo elemento $a$ de $A$.
\end{Nom}

Uma vez que definimos o conceito de ``elemento neutro'', estamos em
condições de definir o conceito de ``elemento simétrico''. Vale
ressaltar que são justamente esses conceitos, mais o da associatividade,
que formaram o conceito de ``grupo'', exibido logo a seguir.

O leitor deve ainda notar que, embora a
nomenclatura\xspace\ref{nomsimetrizavel} exija que, para um elemento $a$
de $A$ ser simetrizável, $a$ precise possuir simétrico em relação a todo
elemento neutro, podemos tranqüilamente definir o mesmo conceito em
relação a um elemento neutro específico.

\begin{Nom}\label{nomsimetrico}
  Sendo um conjunto não vazio
  $A$ munido de uma operação $\ast$ para a qual um
  elemento $e$ de $A$ é um elemento neutro e sendo $a$ e $\simb[elemento
  simétrico de $a$]{\overline{a}}$
  elementos de $A$,
  dizemos que $\overline{a}$ é o \conceito{elemento simétrico} de
  $a$ em relação a $e$ se e só se
  \begin{equation*}
    a\ast \overline{a} = \overline{a}\ast a = e\MMp
  \end{equation*}
\end{Nom}

\begin{Nom}\label{nomsimetrizavel}
  Dizemos que um elemento $a$ de um conjunto não vazio
  $A$ munido de uma operação
  $\ast$ é \Conceito{simetrizável}{elemento simetrizável} em relação a
  $\ast$ se e só se
  existir
  um elemento simétrico de $a$ em relação a qualquer elemento neutro
  para $\ast$.
\end{Nom}

\begin{Nom}
  Sendo um conjunto não vazio
  $A$ munido de uma operação $\ast$, dizemos que um
  elemento $a$ de $A$ é \Conceito{regular para a operação}{elemento
  regular para uma operação} $\ast$ se, para
  quaisquer que
  sejam $x$ e $y$ elementos de $A$, valem as seguintes condicionais:
  \begin{enumerate}[(i)]
    \item se $a\ast x = a\ast y$ então $x=y$;
    \item se $x\ast a = y\ast a$ então $x=y$.
  \end{enumerate}
\end{Nom}

\section{Grupos e subgrupos}

\subsection{Grupos}

\begin{Def}
  Sendo $G$ um conjunto qualquer e $\ast$ uma operação sobre $G$, dizemos
  que $(G,\ast)$ é um \conceito{grupo} se e somente se:
  \begin{enumerate}[(i)]
    \item a operação $\ast$ é associativa;
    \item existe algum elemento neutro em $G$ para a operação
      $\ast$;
    \item todo elemento de $G$ é simetrizável.
  \end{enumerate}
\end{Def}

\begin{Obs}
  Se $(G,\ast)$ é um grupo então $G$ não é vazio, já que precisa haver
  em $G$ pelo menos um elemento neutro para $\ast$.
\end{Obs}

\begin{Propr}\label{proprunicoe}
  A existência de elemento neutro em um grupo $(G,\ast)$ é única.
\end{Propr}

\begin{dem}
  Sejam $e$ e $f$ elementos neutros de um grupo $(G,\ast)$. Queremos,
  então, demonstrar que $e=f$. Como $e$ é um elemento de $G$ e $f$ é um
  elemento neutro, temos que
  \begin{equation*}
    e = e\ast f = f\ast e\MMp
  \end{equation*}
  Mas, como $f$ é um elemento de $G$ e $e$ é um elemento neutro, temos
  também que
  \begin{equation*}
    f\ast e = f\MMp
  \end{equation*}
  Portanto, $e = f$, como queríamos demonstrar.
\end{dem}

\begin{Propr}
  A existência de simétrico de um elemento de $G$ em relação ao
  elemento neutro de um grupo $(G,\ast )$ é única.
\end{Propr}

\begin{dem}
  Seja $e$ o elemento neutro de um grupo $(G,\ast )$ e seja $a$ um
  elemento
  qualquer de $G$. Sejam $\overline{a}_1$ e $\overline{a}_2$ simétricos
  de $a$ em relação a $e$. Queremos, então, mostrar que
  $\overline{a}_1 = \overline{a}_2$. Sabemos que
  \begin{equation}\label{a1EQa1barESTRe}
    \overline{a}_1 = \overline{a}_1\ast e
  \end{equation}
  e, já que $\overline{a}_2$ é simétrico de $a$, que
  \begin{equation}\label{eEQaESTRa2bar}
    e = a\ast \overline{a}_2\MMp
  \end{equation}
  Substituindo (\ref{eEQaESTRa2bar}) em (\ref{a1EQa1barESTRe}),
  temos que
  \begin{equation*}
    \overline{a}_1 = \overline{a}_1\ast (a\ast \overline{a}_2)
  \end{equation*}
  e, portanto, dos axiomas da definição de grupo, que
  \begin{equation*}
    \begin{aligned}
      \overline{a}_1 &= (\overline{a}_1\ast a)\ast \overline{a}_2\\
                     &= e\ast \overline{a}_2\\
                     &= \overline{a}_2\MMv
    \end{aligned}
  \end{equation*}
  como queríamos demonstrar.
\end{dem}

\begin{Propr}
  O simétrico do elemento neutro $e$ de um grupo é próprio elemento
  $e$.
\end{Propr}

\begin{dem}
  Seja $\overline{e}$ o simétrico do elemento neutro $e$ de um grupo
  $(G,\ast )$. Queremos mostrar que $\overline{e}=e$. Sabemos, por $e$ ser
  elemento neutro, que
  \begin{equation*}
    \overline{e} = \overline{e}\ast e\MMp
  \end{equation*}
  Por outro lado, por $\overline{e}$ ser o simétrico de $e$, temos que
  \begin{equation*}
    \overline{e}\ast e = e
  \end{equation*}
  e, portanto, que
  \begin{equation*}
    \overline{e} = e\MMv
  \end{equation*}
  como queríamos demonstrar.
\end{dem}

\begin{Propr}
  Sendo $a$ um elemento de um grupo $(G,\ast )$,
  \begin{equation*}
    \overline{\overline{a}} = a\MMp
  \end{equation*}
\end{Propr}

\begin{dem}
  Sabemos que
  \begin{equation*}
    \overline{\overline{a}} = \overline{\overline{a}}\ast e
  \end{equation*}
  e, como $e = \overline{a}\ast a$, que
  \begin{equation*}
    \overline{\overline{a}} =
      \overline{\overline{a}}\ast \bigl(\overline{a}\ast a\bigr)\MMp
  \end{equation*}
  Portanto, dos axiomas da definição de grupo, temos que
  \begin{equation*}
    \begin{aligned}
      \overline{\overline{a}}
        &= \bigl(\overline{\overline{a}}\ast \overline{a}\bigr)\ast a \\
        &= e\ast a \\
        &= a\MMv
    \end{aligned}
  \end{equation*}
  como queríamos demonstrar.
\end{dem}

\begin{Propr}\label{psmorg}
  Sendo $(G,\ast )$ um grupo e $a$ e $b$ elementos de $G$,
  \begin{equation*}
    \overline{a\ast b} = \overline{b}\ast \overline{a}\MMp
  \end{equation*}
\end{Propr}

\begin{dem}
  Vamos mostrar que:
  \begin{align}
    (a\ast b)\ast (\overline{b}\ast \overline{a}) &=
    e\MMpv\label{psmorgi}  \\
    (\overline{b}\ast \overline{a})\ast (a\ast b) &=
    e\MMp\label{psmorgii}
  \end{align}

  Dos axiomas da definição de grupo, temos que
  \begin{equation*}
    \begin{aligned}
      (a\ast b)\ast (\overline{b}\ast \overline{a})
        &= \bigl((a\ast b)\ast \overline{b}\bigr)\ast \overline{a} \\
        &= \bigl(a\ast (b\ast \overline{b})\bigr)\ast \overline{a} \\
        &= (a\ast e)\ast \overline{a} \\
        &= a\ast \overline{a} \\
        &= e
    \end{aligned}
  \end{equation*}
  e, portanto, mostramos (\ref{psmorgi}). Analogamente,
  \begin{equation*}
    \begin{aligned}
      (\overline{b}\ast \overline{a})\ast (a\ast b)
        &= \bigl((\overline{b}\ast \overline{a})\ast a\bigr)\ast b \\
        &= \bigl(\overline{b}\ast (\overline{a}\ast a)\bigr)\ast b \\
        &= (\overline{b}\ast e)\ast b \\
        &= \overline{b}\ast b \\
        &= e\MMv
    \end{aligned}
  \end{equation*}
  e, portanto, mostramos (\ref{psmorgii}) e concluímos a demonstração.
\end{dem}

\begin{Propr}\label{regulargrupo}
  Em um grupo $(G,\ast )$, todo elemento de $G$ é regular para a
  operação
  $\ast $.
\end{Propr}

\begin{dem}
  Sejam $a$, $x$ e $y$ elementos de
  $G$. Queremos mostrar que:
  \begin{enumerate}[({\ref{regulargrupo}}.i)]
    \item se $a\ast x = a\ast y$ então $x=y$;
    \item se $x\ast a = y\ast a$ então $x=y$.
  \end{enumerate}

  Suponhamos primeiramente que $a\ast x = a\ast y$. Assim, temos que
  \begin{equation*}
    \overline{a}\ast (a\ast x) = \overline{a}\ast (a\ast y)
  \end{equation*}
  e, portanto, que
  \begin{equation*}
    (\overline{a}\ast a)\ast x = (\overline{a}\ast a)\ast y
  \end{equation*}
  e, conseqüentemente, que
  \begin{equation*}
    x = y\MMp
  \end{equation*}

  Agora valha que $x\ast a = y\ast a$. Dessarte,
  \begin{equation*}
    (x\ast a)\ast \overline{a} = (y\ast a)\ast \overline{a}\MMpv
  \end{equation*}
  logo,
  \begin{equation*}
    x\ast (a\ast \overline{a}) = y\ast (a\ast \overline{a})
  \end{equation*}
  e, por fim,
  \begin{equation*}
    x = y\MMv
  \end{equation*}
  o que encerra a demonstração.
\end{dem}

\begin{Propr}\label{propreqgrupo}
  Em um grupo $(G,\ast )$, sendo $a$ e $b$ elementos de $G$, a equação
  \begin{equation}\label{aXb}
    a\ast x = b
  \end{equation}
  tem conjunto solução unitário, constituído do elemento
  $(\overline{a}\ast b)$, assim como a equação
  \begin{equation}\label{Xab}
    x\ast a = b\MMv
  \end{equation}
  cuja única solução é $(b\ast \overline{a})$.
\end{Propr}

\begin{dem}
  Vamos mostrar que:
  \begin{enumerate}[({\ref{propreqgrupo}}.i)]
    \item a solução da equação\xspace\ref{aXb} é única;
    \item $(\overline{a}\ast b)$ é solução da equação\xspace\ref{aXb};
    \item a solução da equação\xspace\ref{Xab} é única;
    \item $(b\ast \overline{a})$ é solução da equação\xspace\ref{Xab}.
  \end{enumerate}

  Sejam $x_1$ e $x_2$ soluções da equação\xspace\ref{aXb}. Assim,
  $a\ast x_1=a\ast x_2$ e, portanto, da
  propriedade\xspace\ref{regulargrupo},
  $x_1=x_2$.

  Dos axiomas da definição de grupo, é verdade que
  \begin{equation*}
    a\ast (\overline{a}\ast b) = (a\ast \overline{a})\ast b
    = e\ast b = b\MMp
  \end{equation*}
  Portanto, $(\overline{a}\ast b)$ é solução da equação\xspace\ref{aXb}.

  Sejam $x_1$ e $x_2$ soluções da equação\xspace\ref{Xab}. Dessarte,
  $x_1\ast a=x_2\ast a$ e, conseqüentemente, da
  propriedade\xspace\ref{regulargrupo},
  $x_1=x_2$.

  Dos axiomas da definição de grupo, é verdade que
  \begin{equation*}
    (b\ast \overline{a})\ast a = b\ast (a\ast \overline{a})
    = b\ast e = b\MMp
  \end{equation*}
  Logo, $(b\ast \overline{a})$ é solução da equação\xspace\ref{Xab}.
\end{dem}

\begin{Not}\label{estrelatorio}
  Sendo $e$ o elemento neutro de um grupo $(G,\ast )$, $m$ e $n$ números
  inteiros e $a$ uma injeção de $[m..n]$ em $G$, denotando-se $a(j)$ por
  $a_j$,
  \begin{equation*}
    \simb[$a_m\ast a_{m+1}\ast\dotsb\ast a_n$]{\bigsqcup_{j=m}^na_j} =
    \left\{
      \begin{aligned}
        &e\MMv\quad&\text{se $m>n$;}\\
        &a_m\ast \biggl(\bigsqcup_{j=m+1}^na_j\biggr)
          \MMv\quad&\text{se $m\leq n$.}
      \end{aligned}
    \right.
  \end{equation*}
\end{Not}

\begin{Teo}
  Sendo $e$ o elemento neutro de um grupo $(G,\ast )$, $m$ e $n$ números
  inteiros e
  $a$ uma injeção de $[m..n]$ em $G$, denotando-se $a(j)$ por
  $a_j$,
  \begin{equation*}
    \overline{\bigsqcup_{j=m}^na_j} =
    \bigsqcup_{j=m}^n\overline{a_{n-j+m}}\MMp
  \end{equation*}
\end{Teo}

\begin{dem}
  Se
  $m>n$, a igualdade se mostra trivialmente na medida em que tanto
  \begin{equation*}
    \overline{\bigsqcup_{j=m}^na_j} = e
  \end{equation*}
  quanto
  \begin{equation*}
    \bigsqcup_{j=m}^n\overline{a_{n-j+m}} = e\MMp
  \end{equation*}
  Suponhamos, então, que $m\leq n$. Se $m-n=0$, temos que
  \begin{equation*}
    \begin{aligned}
      \overline{\bigsqcup_{j=m}^na_j} &=
        \overline{a_m\ast \biggl(\bigsqcup_{j=m+1}^na_j\biggr)}\\
        &= \overline{a_m\ast e} \\
        &= \overline{a_m} \\
        &= \overline{a_m}\ast \overline{e} \\ &=
        \overline{a_m}\ast \biggl(
           \bigsqcup_{j=m+1}^n\overline{a_{n-j+(m+1)}}
        \biggr)\\
        &= \bigsqcup_{j=m}^n\overline{a_{n-j+m}}\MMp
    \end{aligned}
  \end{equation*}
  Seja $k$ um número natural. Por indução em $m-n$, suponhamos que, para
  dois números
  inteiros $r$ e $s$, sempre que $r-s\in[0..k]$ valha que
  \begin{equation*}
    \overline{\bigsqcup_{j=r}^sa_j} =
    \bigsqcup_{j=r}^s\overline{a_{s-j+r}}\MMp
  \end{equation*}
  Queremos afinal mostrar que se $m-n=k+1$ então
  \begin{equation*}
    \overline{\bigsqcup_{j=m}^na_j} =
    \bigsqcup_{j=m}^n\overline{a_{n-j+m}}\MMp
  \end{equation*}
  Da notação\xspace\ref{estrelatorio} e da
  propriedade\xspace\ref{psmorg},
  \begin{equation*}
    \overline{\bigsqcup_{j=m}^na_j} =
    \overline{a_m\ast \biggl(\bigsqcup_{j=m+1}^na_j\biggr)} =
    \overline{\biggl(\bigsqcup_{j=m+1}^na_j\biggr)}\ast
    \overline{a_m}\MMv
  \end{equation*}
  e, da hipótese da indução, como $n-(m+1)\in[0..k]$,
  \begin{equation*}
    \overline{\bigsqcup_{j=m}^na_j} =
    \biggl(\bigsqcup_{j=m+1}^n\overline{a_{n-j+(m+1)}}\biggr)
      \ast \overline{a_m} =
    \bigsqcup_{j=m}^n\overline{a_{n-j+m}}\MMv
  \end{equation*}
  como queríamos demonstrar.
\end{dem}

\subsection{Grupos abelianos}

\begin{Def}
  Dizemos que um grupo $(G,\ast )$ é \Conceito{abeliano}{grupo abeliano}
  (ou
  \Conceito{comutativo}{grupo comutativo}) se e somente se a operação
  $\ast $ é comutativa.
\end{Def}

\subsection{Subgrupos}

\begin{Def}
  De um grupo $(G,\ast )$ dizemos que um subconjunto não-vazio $H$ de $G$ é
  um
  \conceito{subgrupo} e escrevemos
  \begin{equation*}
    \simb[propriedade de subgrupo de $H$ em relação a
    $(G,\ast)$]{H\subseteq (G,\ast )}
  \end{equation*}
  se e somente se:
  \begin{enumerate}[(i)]
    \item $H$ é fechado para a operação $\ast$;
    \item $(H,\ast_H)$ também for é grupo.
  \end{enumerate}
\end{Def}

\begin{Teo}\label{eqsubgrupo}
  Sendo $e$ o elemento neutro de um grupo $(G,\ast )$ e $H$ um
  subconjunto
  não-vazio de $G$, $H$ é um subgrupo de $(G,\ast )$ se e somente se
  $a\ast \overline{b}\in H$ para todo $a$ e todo $b$ elementos de $H$.
\end{Teo}

\begin{dem}
  Vamos mostrar que:
  \begin{enumerate}[({\ref{eqsubgrupo}}.i)]
    \item\label{teosubgrida} se $H$ é um subgrupo de $(G,\ast )$ então
    $a\ast \overline{b}\in H$
    para todo $a$ e todo $b$ elementos de $H$;
    \item\label{teosubfrida} se, para todo $a$ e todo $b$ elementos de
    $H$, $a\ast \overline{b}\in H$ então $H$ é um subgrupo de $(G,\ast )$.
  \end{enumerate}

  Suponhamos inicialmente que $H$ seja um subgrupo de $(G,\ast )$ e
  indiquemos por $e_H$ o elemento neutro de $(H,\ast _H)$. Como
  \begin{equation*}
    e_H\ast e_H = e_H\ast_H e_H = e_H = e_H\ast e\MMv
  \end{equation*}
  temos, da propriedade\xspace\ref{regulargrupo}, que
  \begin{equation*}
    e_H = e\MMp
  \end{equation*}
  Seja $x$ um elemento qualquer de $H$ e indiquemos por $\overline{x}_H$
  o simétrico de $x$ em $(H,\ast _H)$. Como
  \begin{equation*}
    \overline{x}_H\ast x =
    \overline{x}_H\ast_H x = e_H = e = \overline{x}\ast x\MMv
  \end{equation*}
  temos, novamente da propriedade\xspace\ref{regulargrupo}, que
  \begin{equation*}
    \overline{x}_H = \overline{x}\MMp
  \end{equation*}
  Por fim, tomemos $a$ e $b$ elementos de $H$. Como $(H,\ast _H)$ é um
  grupo, $a\ast_H \overline{b}_H\in H$. Entretanto, como mostramos,
  $\overline{b}_H=\overline{b}$. Logo, $a\ast_H
  \overline{b} = a\ast \overline{b}\in H$, e, assim,
  demonstramos (\ref{teosubgrida}).

  Suponhamos agora que, para quaisquer $a$ e $b$ elementos de $H$,
  $a\ast \overline{b}\in H$ e demonstremos que $H$ é um subgrupo de
  $(G,\ast )$. Uma vez que
  $H$ não é vazio, tomemos um elemento $x_0$ de
  $H$. Assim, temos da hipótese que
  \begin{equation*}
    x_0\ast \overline{x_0}=e\in H\MMp
  \end{equation*} Mais
  do que isso,
  temos utilizando novamente a hipótese também que, para todo $x\in
  H$,
  \begin{equation*}
    e\ast \overline{x} = \overline{x}\in H\MMp
  \end{equation*}
  Sejam $a$ e $b$ elementos de $H$. De acordo com o que
  acabamos de mostrar, $\overline{b}$ também pertence a $H$. Utilizando
  mais uma vez a hipótese, notificamo-nos de que
  \begin{equation*}
    a\ast \overline{\overline{b}} = a\ast b\in H
  \end{equation*}
  e, portanto, de que $\ast$ é fechada para $H$. Como, valendo-nos de
  argumentos já utilizados nesta demonstração,
  $e=e_H\in H$, e
  $\overline{x}=\overline{x}_H\in H$ para todo $x\in H$, e como a
  associatividade da operação $\ast_H$ segue da associatividade de
  $\ast$
  concluímos a demonstração de (\ref{teosubfrida}).
\end{dem}

\subsection{Classes laterais}

\begin{Not}\label{notclasselateral}
  Sendo $H$ um subgrupo de um grupo
  $(G,\ast)$ e $a$ e $b$ elementos quaisquer de
  $G$, escrevemos
  \begin{equation*}
    \simb[relação de congruência à direita entre $a$ e $b$ módulo
    $H$]{\congmodright{a}{b}{H}}
  \end{equation*}
  para denotar que $\overline{a}\ast b\in H$ e
  \begin{equation*}
    \simb[relação de congruência à esquerda entre $a$ e $b$ módulo
    $H$]{\congmodleft{a}{b}{H}}
  \end{equation*}
  para denotar que $a\ast \overline{b}\in H$.
\end{Not}

\begin{Teo}
  Sendo $H$ um subgrupo de um grupo
  $(G,\ast)$, as relações definidas na
  notação\xspace\ref{notclasselateral} são relações de equivalência.
\end{Teo}

\begin{dem}
  É imediato que, para
  qualquer elemento $x$ de $G$,
  \begin{equation*}
    \begin{aligned}
      \congmodrighta{x}{x}{H}\qquad\text{e}\\
      \congmodlefta{x}{x}{H}\MMv
    \end{aligned}
  \end{equation*}
  já que, do teorema\xspace\ref{eqsubgrupo},
  $x\ast\overline{x}=e\in H$ e, por conseqüência,
  $e\ast\overline{(x\ast\overline{x})}=\overline{x}\ast x\in H$.
  Sejam $x_1$, $x_2$, $y_1$ e $y_2$
  elementos de $G$ e suponhamos que
  \begin{equation*}
    \begin{aligned}
      \congmodrighta{x_1}{x_2}{H}\MMv\qquad\text{e}\\
      \congmodlefta{y_1}{y_2}{H}\MMp
    \end{aligned}
  \end{equation*}
  Como $\overline{x_1}\ast x_2\in H$ e $y_1\ast\overline{y_2}\in H$
  e como $H$ é
  um subgrupo, ambos $\overline{\overline{x_1}\ast x_2}=
  \overline{x_2}\ast x_1$
  e $\overline{y_1\ast\overline{y_2}}=
  y_2\ast\overline{y_1}$ pertencem a $H$, e, portanto,
  \begin{equation*}
    \begin{aligned}
      \congmodrighta{x_2}{x_1}{H}\MMv\qquad\text{e}\\
      \congmodlefta{y_2}{y_1}{H}\MMp
    \end{aligned}
  \end{equation*}

  Resta-nos ainda mostrar a transitividade dessas relações. Para tanto,
  tomemos $a_1$, $a_2$, $a_3$, $b_1$, $b_2$ e $b_3$ elementos de $G$ e
  suponhamos que:
  \begin{equation*}
    \begin{aligned}
      \congmodrighta{a_1}{a_2}{H}\MMpv\\
      \congmodrighta{a_2}{a_3}{H}\MMpv\\
      \congmodlefta{b_1}{b_2}{H}\MMpv\\
      \congmodlefta{b_2}{b_3}{H}\MMp\\
    \end{aligned}
  \end{equation*}
  Portanto, $\overline{a_1}\ast a_2$, $\overline{a_2}\ast a_3$,
  $b_1\ast\overline b_2$ e $b_2\ast\overline b_3$ são elementos de $H$
  e, como $H$ é um subgrupo, também são elementos de $H$
  \begin{equation*}
    \begin{aligned}
      (\overline{a_1}\ast a_2)\ast (\overline{a_2}\ast a_3)
      &= a_1\ast a_3\MMv\qquad\text{e}\\
      (b_1\ast\overline b_2)\ast (b_2\ast\overline b_3)
      &= b_1\ast b_3\MMp
    \end{aligned}
  \end{equation*}
  Finalmente,
  \begin{equation*}
    \begin{aligned}
      \congmodrighta{a_1}{a_3}{H}\MMv\qquad\text{e}\\
      \congmodlefta{b_1}{b_3}{H}\MMv
    \end{aligned}
  \end{equation*}
  como queríamos mostrar.
\end{dem}

\begin{Def}\label{defclasselateral}
  Sendo $H$ um subgrupo de um grupo
  $(G,\ast)$ e $a$ um elemento de $G$,
  a \conceito{classe lateral à direita} de $a$ módulo $H$,
  denotada por $\simb[classe lateral à direita de $a$ módulo $H$]{a\ast
  H}$, é a classe de equivalência de $a$ pela relação
  $\sim$ módulo $H$.
  Analogamente, a \conceito{classe lateral à esquerda} de $a$
  módulo $H$, denotada por $\simb[classe lateral à esquerda de $a$
  módulo $H$]{H\ast a}$, é a classe de equivalência de $a$
  pela relação $\backsim$ módulo $H$.
\end{Def}

\begin{Propr}
  Sendo $H$ um subgrupo de um grupo
  $(G,\ast)$ e $a$ um elemento de $G$,
  \begin{equation*}
    \begin{aligned}
      a\ast H &= \cjpp{a\ast h}{h\in H} \MMv\qquad\text{e}\\
      H\ast a &= \cjpp{h\ast a}{h\in H}\MMp
    \end{aligned}
  \end{equation*}
\end{Propr}

\begin{dem}
  Tomemos $x\in(a\ast H)$ e $y\in(H\ast a)$. Assim, da
  definição\xspace\ref{defcleq}, $\overline{x}\ast a = h_1$
  e $y\ast\overline{a} = h_2$ pertencem a $H$. Portanto,
  $x = a\ast\overline{h_1}$, $y = h_2\ast a$, e, conseqüentemente,
  $x\in\cjpp{a\ast h}{h\in H}$ e $y\in\cjpp{h\ast a}{h\in H}$, o que nos
  leva a concluir que
  \begin{equation*}
    \begin{aligned}
      a\ast H &\subseteq \cjpp{a\ast h}{h\in H} \qquad\text{e que}\\
      H\ast a &\subseteq \cjpp{h\ast a}{h\in H}\MMp
    \end{aligned}
  \end{equation*}
  Por outro lado, tomemos um elemento $h$ de $H$. Sabemos que
  $(\overline{a\ast h})\ast a = \overline{h}$
  e, portanto, como $H$ é um subgrupo, que
  \begin{equation*}
    \congmodright{(a\ast h)}{a}{H}\MMp
  \end{equation*}
  Da mesma forma, é verdade que
  $(h\ast a)\ast \overline{a} = h$
  e, dessarte, que
  \begin{equation*}
    \congmodleft{(h\ast a)}{a}{H}\MMp
  \end{equation*}
  Logo, $(a\ast h)\in (a\ast H)$ e $(h\ast a)\in (H\ast a)$, e,
  finalmente,
  \begin{equation*}
    \begin{aligned}
      \cjpp{a\ast h}{h\in H} &\subseteq a\ast H \MMv\qquad\text{e}\\
      \cjpp{h\ast a}{h\in H} &\subseteq H\ast a\MMp
    \end{aligned}
  \end{equation*}
\end{dem}

\begin{Cor}\label{coraHsimeqHa}
  Sendo $H$ um subgrupo de um grupo
  $(G,\ast)$ e $a$ um elemento de $G$, $a\ast H\simeq H\ast a$.
\end{Cor}

\begin{dem}
  Seja $\funcao{f}{a\ast H}{H\ast a}$ a função definida por
  \begin{equation*}
    f(x) = h_x\ast a\MMv
  \end{equation*}
  sendo $h_x$ o elemento de $H$ tal que $x=a\ast h_x$. Sejam $x_1$ e
  $x_2$ elementos distintos de $a\ast H$. Da
  propriedade\xspace\ref{regulargrupo}, $h_{x_1}\neq h_{x_2}$, sendo
  $h_{x_1}$ o elemento de $H$ tal que $x_1=a\ast h_{x_1}$ e
  $h_{x_2}$ o elemento de $H$ tal que $x_2=a\ast h_{x_2}$. Logo, também
  por
  causa da propriedade\xspace\ref{regulargrupo}, temos
  que $h_{x_1}\ast a\neq h_{x_2}\ast a$ e, conseqüentemente, que
  $f(x_1)\neq f(x_2)$, o que nos leva a concluir que $f$ é injetiva.

  Para mostrarmos que $f$ é sobrejetiva, é suficiente que mostremos que
  $H\ast a\subseteq f(a\ast H)$. Para tanto, tomemos um elemento $y$ de
  $H\ast a$, sendo $h_y$ o elemento de $H$ tal que $y=h_y\ast a$, e
  observemos que $f(a\ast h_y) = h_y\ast a$. Assim, $y\in f(a\ast H)$ e,
  dessarte, $f$ também é sobrejetiva.
\end{dem}

\begin{Cor}\label{coraHsimeqbH}
  Sendo $H$ um subgrupo de um grupo
  $(G,\ast)$ e $a$ e $b$ elemento quaisquer de $G$, $a\ast H\simeq b\ast
  H$.
\end{Cor}

\begin{dem}
  Seja $\funcao{f}{a\ast H}{b\ast H}$ a função definida por
  \begin{equation*}
    f(x) = b\ast h_x\MMv
  \end{equation*}
  sendo $h_x$ o elemento de $H$ tal que $x=a\ast h_x$. Sejam $x_1$ e
  $x_2$ elementos distintos de $a\ast H$. Da
  propriedade\xspace\ref{regulargrupo}, $h_{x_1}\neq h_{x_2}$, sendo
  $h_{x_1}$ o elemento de $H$ tal que $x_1=a\ast h_{x_1}$ e
  $h_{x_2}$ o elemento de $H$ tal que $x_2=a\ast h_{x_2}$. Logo, também
  por
  causa da propriedade\xspace\ref{regulargrupo}, temos
  que $b\ast h_{x_1}\neq b\ast h_{x_2}$ e, conseqüentemente, que
  $f(x_1)\neq f(x_2)$, o que nos leva a concluir que $f$ é injetiva.

  Para mostrarmos que $f$ é sobrejetiva, é suficiente que mostremos que
  $b\ast H\subseteq f(a\ast H)$. Para tanto, tomemos um elemento $y$ de
  $b\ast H$, sendo $h_y$ o elemento de $H$ tal que $y=b\ast h_y$, e
  observemos que $f(a\ast h_y) = b\ast h_y$. Assim, $y\in f(a\ast H)$ e,
  dessarte, $f$ também é sobrejetiva.
\end{dem}

\subsection{Grupos finitos}

\begin{Nom}
  Dizemos que um grupo $(G,\ast )$ é \Conceito{finito}{grupo finito}
  quando
  e só quando $G$ é finito. Ademais, dizemos que $\cardi{G}$ é a
  \Conceito{ordem}{ordem de um grupo finito} de
  $(G,\ast)$. Analogamente, se $H$ é um subgrupo de $(G,\ast)$ então
  $\cardi{H}$ é a
  \Conceito{ordem}{ordem de um subgrupo de um grupo finito} de $H$.
\end{Nom}

\begin{Def}
  Sendo $H$ um subgrupo de um grupo finito $(G,\ast)$, chamamos de
  \Conceito{índice}{índice de um subgrupo num grupo} de $H$ em $G$,
  e denotamos por $\simb[cardinalidade do conjunto quociente
  de $G$ por $\sim$ ou $\backsim$ módulo
  $H$]{[G:H]}$, a
  cardinalidade do conjunto quociente de $G$ por qualquer uma das
  relações $\sim$ e $\backsim$ módulo $H$.
\end{Def}

\begin{Obs}
  Note-se que a arbitrariedade na escolha da relação módulo $H$ tomada
  deve-se
  sobretudo ao corolário\xspace\ref{coraHsimeqHa}.
\end{Obs}

\begin{Teo}[Teorema de Lagrange]\label{teolagrange}
  \index{teorema de Lagrange}
  \mbox{}\marginpar%
  [\raggedright\hspace{0pt}\footnotesize %
    teorema de Lagrange\\\vspace{\baselineskip}]%
  {\raggedleft\hspace{0pt}\footnotesize %
    teorema de Lagrange\\\vspace{\baselineskip}}
  Sendo $(G,\ast)$ um grupo finito, a ordem de qualquer subgrupo de
  $(G,\ast)$ divide a ordem de $(G,\ast)$.
\end{Teo}

\begin{dem}
  Seja $H$ um subgrupo de $(G,\ast)$.
  Como $G$ é finito, o conjunto $Q$ quociente de $G$ pela relação $\sim$
  módulo $H$ também é finito e, portanto, $Q$ pode ser escrito como:
  \begin{equation*}
    Q = \bigcup_{j=1}^{[G:H]}\cj{a_j\ast H}\MMp
  \end{equation*}
  Entretanto, do teorema\xspace\ref{teocquoc}, $Q$ é uma partição de $G$
  e, conseqüentemente,
  \begin{equation*}
    G = \bigcup_{j=1}^{[G:H]} a_j\ast H\MMp
  \end{equation*}
  Também por $Q$ ser uma partição de $G$ concluímos que se $k\neq\ell $,
  sendo $\ell$ e $k$ elementos de $[[G:H]]$, então $a_k\ast H$ e 
  $a_{\ell}\ast H$ são conjuntos disjuntos. Assim,
  \begin{equation*}
    \cardi{G} = \sum_{j=1}^{[G:H]} \cardi{a_j\ast H}
  \end{equation*}
  e, do corolário\xspace\ref{coraHsimeqbH}, como, para todo $j\in[[G:H]]$,
  $a_j\ast H\simeq e\ast H = H$,
  \begin{equation*}
    \cardi{G} = \sum_{j=1}^{[G:H]} \cardi{H}\MMp
  \end{equation*}
  Finalmente,
  \begin{equation*}
    \cardi{G} = [G:H]\cardi{H}\MMv
  \end{equation*}
  e, da definição\xspace\ref{defdivide}, como $[G:H]$ é um número inteiro,
  $\cardi{H}\divd \cardi{G}$, como queríamos mostrar.
\end{dem}

\subsection{Grupos cíclicos}

O conceito que apresentamos a seguir trata de um caso particular de
grupo. Pelo princípio da indução, podemos somar $1$ sucessivamente
partindo de $0$ e chegar em qualquer número natural. Entretanto, não é
possível construir um processo semelhante para o grupo dos racionais,
por exemplo. O motivo pelo qual isso ocorre é que os racionais não
constituem o que é chamado de ``grupo cíclico''.

\begin{Def}\label{defiteracao}
  Seja $(G,\ast)$ um grupo de elemento neutro $e$,
  $a$ um elemento qualquer de $G$ e
  $n$ um número natural. A $n$-ésima \Conceito{iteração}{iteração de uma
  operação} de $\ast$ sobre $a$ é o elemento de $G$ definido por:
  \begin{equation*}
    \simb[$n$-ésima iteração de $\ast$ sobre $a$]{a^{\ast n}} = \left\{
    \begin{aligned}
      &\bigl(a^{\ast(n-1)}\bigr)\ast a\MMv&\quad&\text{se $n>0$;}\\
      &e\MMv&\quad&\text{caso contrário.}
    \end{aligned}
    \right.
  \end{equation*}
\end{Def}

\begin{Propr}\label{proprpotalternativa}
  Sendo  $(G,\ast)$ um grupo de elemento neutro $e$,
  $a$ um elemento qualquer de $G$ e
  $n$ um número natural,
  \begin{equation*}
    a^{\ast n} = \left\{
    \begin{aligned}
      &a\ast \bigl(a^{\ast(n-1)}\bigr)\MMv&\quad&\text{se $n>0$;}\\
      &e\MMv&\quad&\text{caso contrário.}
    \end{aligned}
    \right.
  \end{equation*}
\end{Propr}

\begin{dem}
  É imediato que $a^{\ast 0} = e$. Por indução em $n$, suponhamos que,
  sendo $k$ um natural, para todo $\ell\in[0..k]$ vale que
  \begin{equation*}
    a^{\ast \ell} = \left\{
    \begin{aligned}
      &a\ast
      \bigl(a^{\ast(\ell-1)}\bigr)\MMv&\quad&\text{se $\ell>0$;}\\
      &e\MMv&\quad&\text{caso contrário.}
    \end{aligned}
    \right.
  \end{equation*}
  Como $k+1>0$, queremos, então, mostrar que
  $a^{\ast (k+1)} = a\ast\bigl(a^{\ast((k+1)-1)}\bigr)$. Ora, da
  definição\xspace\ref{defiteracao},
  \begin{equation*}
    a^{\ast (k+1)} = \bigl(a^{\ast((k+1)-1)}\bigr)\ast a\MMp
  \end{equation*}
  Assim, uma vez que $(k+1)-1=k\in[0..k]$, se $k=0$ então $a^{\ast k}=e$
  e, portanto,
  \begin{equation*}
    \begin{aligned}
      a^{\ast (k+1)} &= e\ast a\\
                     &= a\\
                     &= a\ast e\\
                     &= a\ast\bigl(a^{\ast((k+1)-1)}\bigr)\MMv
    \end{aligned}
  \end{equation*}
  como queríamos mostrar. Se, entretanto, $k>0$ então, da hipótese da
  indução e do axioma da associatividade,
  \begin{equation*}
    \begin{aligned}
      a^{\ast (k+1)} &= \Bigl(a\ast\bigl(a^{\ast(k-1)}\bigr)\Bigr)
                        \ast a\\
             &= a\ast\Bigl(\bigl(a^{\ast(k-1)}\bigr)\ast a\Bigr)\MMv
    \end{aligned}
  \end{equation*}
  e, portanto, novamente da definição\xspace\ref{defiteracao},
  \begin{equation*}
    a^{\ast (k+1)} = a\ast\bigl(a^{\ast((k-1)+1)}\bigr)\MMv
  \end{equation*}
  como queríamos mostrar.
\end{dem}

\begin{Propr}\label{apotm-apotn-apotmn}
  Sendo $(G,\ast)$ um grupo, $a$ um elemento de $G$ e
  $n$ e $m$ números naturais,
  \begin{equation*}
    a^{\ast m}\ast a^{\ast n} = a^{\ast (m+n)}\MMp
  \end{equation*}
\end{Propr}

\begin{dem}
  É imediato que $a^{\ast m}\ast a^{\ast 0} = a^{\ast (m+0)}$, já que
  $a^{\ast 0} = e$, sendo $e$ o elemento neutro do grupo. Seja $k$ um
  número natural qualquer. Por indução em $n$, supondo que vale que
  $a^{\ast m}\ast a^{\ast\ell} = a^{\ast (m+\ell)}$ para todo
  $\ell\in[0..k]$,
  o que queremos é mostrar que $a^{\ast m}\ast a^{\ast(k+1)}
  = a^{\ast(m+(k+1))}$. Da
  definição\xspace\ref{defiteracao},
  como $k+1>0$,
  \begin{equation*}
    a^{\ast m}\ast a^{\ast(k+1)} = a^{\ast m}\ast(a^{\ast k}\ast a)
    = (a^{\ast m}\ast a^{\ast k})\ast a\MMpv
  \end{equation*}
  portanto, da hipótese da indução, como $k\in[0..k]$,
  \begin{equation*}
    a^{\ast m}\ast a^{\ast(k+1)} = a^{\ast(m+k)}\ast a\MMv
  \end{equation*}
  e, conseqüentemente, novamente da definição\xspace\ref{defiteracao},
  \begin{equation*}
    a^{\ast m}\ast a^{\ast(k+1)} = a^{\ast((m+k)+1)}
    = a^{\ast(m+(k+1))}\MMv
  \end{equation*}
  como queríamos mostrar.
\end{dem}

\begin{Propr}\label{proprpotmnpotmn}
  Sendo $(G,\ast)$ um grupo, $a$ um elemento de $G$ e
  $n$ e $m$ números naturais,
  \begin{equation*}
    (a^{\ast m})^{\ast n} = a^{\ast(mn)}\MMp
  \end{equation*}
\end{Propr}

\begin{dem}
  É imediato que $(a^{\ast m})^{\ast 0} = a^{\ast(m\cdot 0)}$,
  já que $(a^{\ast m})^{\ast 0} = \um = a^{\ast 0}$.
  Seja $k$ um
  número natural qualquer. Por indução em $n$, supondo que vale que
  $(a^{\ast m})^{\ast \ell} = a^{\ast (m\ell)}$ para todo
  $\ell\in[0..k]$,
  o que queremos é mostrar que $(a^{\ast m})^{\ast(k+1)}
  = a^{\ast(m(k+1))}$. Da definição
  de potência, como $k+1>0$,
  \begin{equation*}
    (a^{\ast m})^{\ast(k+1)} =
    \bigl((a^{\ast m})^{\ast k}\bigr)\ast(a^{\ast m})\MMpv
  \end{equation*}
  portanto, da hipótese da indução, como $k\in[0..k]$,
  \begin{equation*}
    (a^{\ast m})^{\ast(k+1)} = a^{\ast(mk)}\ast a^{\ast m}\MMv
  \end{equation*}
  e, conseqüentemente, da proposição\xspace\ref{apotm-apotn-apotmn},
  \begin{equation*}
    (a^{\ast m})^{\ast(k+1)} = a^{\ast(mk+m)} = a^{\ast(m(k+1))}\MMv
  \end{equation*}
  como queríamos mostrar.
\end{dem}

\begin{Propr}\label{abpotn-apotnbpotn}
  Sendo $(G,\ast)$ um grupo, $a$ e $b$ elementos de $G$ e
  $n$ um  número natural,
  \begin{equation*}
    (a\ast b)^{\ast n} = a^{\ast n}\ast b^{\ast n}\MMp
  \end{equation*}
\end{Propr}

\begin{dem}
  Sabemos que, sendo $e$ o elemento neutro de $G$, $(a\ast b)^{\ast 0} =
  e = e\ast e = a^{\ast 0}\ast b^{\ast 0}$. Seja $k$ um natural e
  suponhamos, por indução em $n$, que $(a\ast b)^{\ast \ell} = a^{\ast
  \ell}\ast b^{\ast \ell}$ para todo $\ell\in[0..k]$. Queremos então
  apenas mostrar que $(a\ast b)^{\ast (k+1)} = a^{\ast
    (k+1)}\ast b^{\ast (k+1)}$. Ora,
  \begin{equation*}
      (a\ast b)^{\ast (k+1)} = (a\ast b)^{\ast k}\ast (a\ast b)\MMv
  \end{equation*}
  e, portanto, da hipótese de indução,
  \begin{equation*}
    \begin{aligned}
      (a\ast b)^{\ast (k+1)}
        &= (a^{\ast k}\ast b^{\ast k})\ast(a\ast b) \\
        &= (a^{\ast k}\ast a)\ast(b^{\ast k}\ast b) \\
        &= a^{\ast (k+1)}\ast b^{\ast (k+1)}\MMv
    \end{aligned}
  \end{equation*}
  como queríamos mostrar.
\end{dem}

\begin{Propr}\label{proproveranoveran}
  Sendo $(G,\ast)$ um grupo, $a$ um elemento de $G$ e
  $n$ um número natural,
  \begin{equation*}
    \overline{a^{\ast n}} = (\overline{a})^{\ast n}\MMp
  \end{equation*}
\end{Propr}

\begin{dem}
  Sabemos que $\overline{a^{\ast 0}} = \overline{e} = e =
  (\overline{a})^{\ast 0}$, sendo $e$ o elemento neutro de $(G,\ast)$.
  Suponhamos, então, por indução em $n$, que, sendo $k$ um número
  natural,
  para todo $\ell\\in[0..k]$ vale que
  \begin{equation*}
    \overline{a^{\ast \ell}} = (\overline{a})^{\ast \ell}\MMp
  \end{equation*}
  Queremos, assim, por causa da nomenclatura\xspace\ref{nomsimetrico},
  apenas mostrar que:
  \begin{enumerate}[({\ref{proproveranoveran}}.i)]
    \item $\bigl(a^{\ast (k+1)}\bigr)\ast
      \bigl((\overline{a})^{\ast (k+1)}\bigr) = e$;
    \item $\bigl((\overline{a})^{\ast (k+1)}\bigr)\ast
      \bigl(a^{\ast (k+1)}\bigr) = e$.
  \end{enumerate}
  Sabemos, da definição\xspace\ref{defiteracao} e da
  propriedade\xspace\ref{proprpotalternativa}, e do axioma da
  associatividade,
  que
  \begin{equation*}
    \begin{aligned}
      \bigl(a^{\ast (k+1)}\bigr)\ast
      \bigl((\overline{a})^{\ast (k+1)}\bigr) &=
      \bigl(a\ast a^{\ast k}\bigr)\ast
      \bigl((\overline{a})^{\ast k}\ast \overline{a}\bigr) \\&=
      a\ast
      \bigl(a^{\ast k}\ast (\overline{a})^{\ast k}\bigr)
      \ast \overline{a}\qquad\text{e que}\\
      \bigl((\overline{a})^{\ast (k+1)}\bigr)\ast
      \bigl(a^{\ast (k+1)}\bigr) &=
      \bigl(\overline{a}\ast (\overline{a})^{\ast k}\bigr)\ast
      \bigl(a^{\ast k}\ast a\bigr) \\&=
      \overline{a}\ast
      \bigl((\overline{a})^{\ast k}\ast a^{\ast k}\bigr)
      \ast a\MMv
    \end{aligned}
  \end{equation*}
  e, portanto, da hipótese da indução, que
  \begin{equation*}
    \begin{aligned}
      \bigl(a^{\ast (k+1)}\bigr)\ast
      \bigl((\overline{a})^{\ast (k+1)}\bigr) &=
      a\ast
      \bigl(a^{\ast k}\ast \overline{a^{\ast k}}\bigr)
      \ast \overline{a} \\&=
      a\ast e\ast\overline{a} \\&=e\qquad\text{e que}\\
      \bigl((\overline{a})^{\ast (k+1)}\bigr)\ast
      \bigl(a^{\ast (k+1)}\bigr) &=
      \overline{a}\ast
      \bigl(\overline{a^{\ast k}}\ast a^{\ast k}\bigr)
      \ast a\MMv\\&=
      \overline{a}\ast e\ast a\\&= e\MMv
    \end{aligned}
  \end{equation*}
  exatamente como queríamos mostrar.
\end{dem}

\begin{Def}
  Dizemos que um grupo $(G,\ast)$ é \Conceito{cíclico}{grupo cíclico} se
  e somente se existe um $a$ em $G$ tal que, para todo $b\in G$, exista
  um $j\in\MMN\setminus\cj{0}$ tal que $a^{\ast n}=b$.
\end{Def}

\begin{Teo}
  Todo grupo cíclico é comutativo.
\end{Teo}

\begin{dem}
  Seja $(G,\ast)$ um grupo cíclico e seja $a$ um elemento de $G$ tal
  que, para todo $b\in G$, exista um $j\in\MMN$ tal que $a^{\ast
  n}=b$. Sejam $x$ e $y$ elementos de $G$ e $j_x$ e $j_y$ números
  naturais tais que $a^{\ast j_x}=x$ e $a^{\ast j_y}=y$. Assim,
  \begin{equation*}
    \begin{aligned}
      x\ast y &= a^{\ast j_x}\ast a^{\ast j_y}\\
              &= a^{\ast(j_x+j_y)}\\
              &= a^{\ast(j_y+j_x)}\\
              &= a^{\ast j_y}\ast a^{\ast j_x}\\
              &= y\ast x\MMv
    \end{aligned}
  \end{equation*}
  como queríamos mostrar.
\end{dem}

\begin{Lem}\label{lemanastoveram}
  Sendo $(G,\ast)$ um grupo, $a$ um elemento de $G$ e
  $n$ e $m$ números naturais,
  \begin{equation*}
    a^{\ast n}\ast \overline{a^{\ast m}} = \left\{
    \begin{aligned}
      &e\MMv&\quad&\text{se $n=m$;}
      &a^{\ast (n-m)}\MMv&\quad&\text{se $n>m$;}
      &a^{\ast (m-n)}\MMv&\quad&\text{se $n<m$.}
    \end{aligned}
    \right.
  \end{equation*}
\end{Lem}

\begin{dem}
  Se $n=m$, a asserção se verifica imediatamente. Se $n>m$, porém,
  então, da propriedade\xspace\ref{apotm-apotn-apotmn},
  \begin{equation*}
    \begin{aligned}
      a^{\ast n}\ast \overline{a^{\ast m}}
        &= \bigl(a^{\ast (n-m)}\ast a^{\ast m}\bigr)
           \ast\overline{a^{\ast m}}\\
        &= a^{\ast (n-m)}\ast\bigl(a^{\ast m}\ast
           \overline{a^{\ast m}}\bigr)\\
        &= a^{\ast (n-m)}\ast e\\
        &= a^{\ast (n-m)}\MMp
    \end{aligned}
  \end{equation*}
  Se $n<m$, por sua vez, então, da
  propriedade\xspace\ref{proproveranoveran},
  \begin{equation*}
    a^{\ast n}\ast \overline{a^{\ast m}}
    = a^{\ast n}\ast(\overline{a})^{\ast m}\MMv
  \end{equation*}
  e, da propriedade\xspace\ref{apotm-apotn-apotmn},
  \begin{equation*}
    \begin{aligned}
      a^{\ast n}\ast \overline{a^{\ast m}}
      &= a^{\ast n}\ast
         \bigl(
         (\overline{a})^{\ast n}\ast (\overline{a})^{\ast (m-n)}
         \bigr)\\
      &= \bigl(a^{\ast n}\ast (\overline{a})^{\ast n}\bigr)
         \ast (\overline{a})^{\ast (m-n)}\MMv
    \end{aligned}
  \end{equation*}
  e, novamente da propriedade\xspace\ref{proproveranoveran},
  \begin{equation*}
    \begin{aligned}
      a^{\ast n}\ast \overline{a^{\ast m}} &=
        \bigl(a^{\ast n}\ast \overline{a^{\ast n}}\bigr)
        \ast \overline{a^{\ast (m-n)}}\\&=
        e\ast \overline{a^{\ast (m-n)}}\\&=
        \overline{a^{\ast (m-n)}}\MMv
    \end{aligned}
  \end{equation*}
  como queríamos mostrar.
\end{dem}

\begin{Lem}\label{lemate}
  Sendo $(G,\ast)$ um grupo finito
  e $a$ um elemento de $G$, existe um $t\in\MMN\setminus\cj{0}$ tal que
  $a^{\ast t} = e$, sendo $e$ o elemento neutro de $(G,\ast)$.
\end{Lem}

\begin{dem}
  Suponhamos que $a\neq e$, já que se $a=e$ então a afirmação se
  verifica trivialmente, e definamos, para todo $n\in\MMN$, os
  conjuntos:
  \begin{equation*}
    A_n = \left\{
    \begin{aligned}
      &\emptyset\MMv&\quad&\text{se $n=0$}\\
      &\cj{a^{\ast n}}\cup A_{n-1}\MMv&\quad&\text{caso contrário.}
    \end{aligned}
    \right.
  \end{equation*}
  É imediato que, para todo $n\in\MMN$, $A_n\subseteq G$, e, como $G$ é
  finito, $\cardi{A_n}\leq \cardi{G}$.
  Suponhamos, então, que não exista inteiro
  positivo $t$ tal que $a^{\ast t} = e$. Assim, não existe inteiro
  positivo $s$ tal que $e\in A_s$. Portanto, para todo $n\in\MMN$,
  $a^{\ast n}\notin A_{n-1}$, pois se $a^{\ast n}\in A_{n-1}$ então
  haveria um inteiro positivo $j<n$ tal que $a^{\ast n}=a^{\ast j}$, e,
  dessarte, $a^{\ast (n-j)}=e$, e, como $n-j\leq n-1$, $e$ pertenceria a
  $A_{n-1}$. Logo, $\cardi{A_0}=0$ e
  $\cardi{A_n}=\cardi{A_{n-1}}+1$ para todo $n>0$, e, por conseqüência,
  $\cardi{A_n}=n$ para todo $n\in\MMN$. Em particular,
  $\cardi{A_{\cardi{G}+1}}=\cardi{G}+1$, o que é um absurdo.
\end{dem}

\begin{Teo}\label{teogeradoasubgrupo}
  Sendo $(G,\ast)$ um grupo finito
  e $a$ um elemento de $G$, o conjunto
  \begin{equation*}
    \simb[subgrupo gerado por $a$]{\gerado{a}} = \cjpp{a^{\ast
    k}}{k\in\MMN\setminus\cj{0}}
  \end{equation*}
  é um subgrupo de $(G,\ast)$, chamado de \conceito{subgrupo
  gerado} por $a$.
\end{Teo}

\begin{dem}
  É claro que $\gerado{a}\neq\emptyset$, pois ao menos $a^{\ast
  1}=a\in\gerado{a}$. Sejam, portanto, $j$ e $k$ inteiros
  positivos. Sabemos, do lema\xspace\ref{lemanastoveram}, que
  \begin{equation*}
    a^{\ast j}\ast \overline{a^{ast k}} =\left\{
    \begin{aligned}
      &e\MMv&\quad&\text{se $j=k$;}
      &a^{\ast (j-k)}\MMv&\quad&\text{se $j>k$;}
      &a^{\ast (k-j)}\MMv&\quad&\text{se $k<j$.}
    \end{aligned}
    \right.
  \end{equation*}
  Logo, para mostrarmos que existe um $\ell\in\MMN\setminus\cj{0}$ tal
  que $a^{\ast \ell}=a^{\ast j}\ast \overline{a^{ast k}}$, e, dessarte,
  concluirmos que $a^{\ast j}\ast \overline{a^{ast k}}\in\gerado{a}$,
  basta que mostremos que existe um $t\in\MMN\setminus\cj{0}$
  tal que $a^{\ast t}=e$, o que é garantido pelo
  lema\xspace\ref{lemate}. Assim, $a^{\ast j}\ast \overline{a^{ast
  k}}\in\gerado{a}$ e, conseqüentemente, do
  teorema\xspace\ref{eqsubgrupo}, $\gerado{a}$ é um subgrupo de
  $(G,\ast)$, como queríamos mostrar.
\end{dem}

\begin{Cor}
  Sendo $(G,\ast)$ um grupo finito
  e $a$ um elemento de $G$, $(\gerado{a},\ast_{\gerado{a}})$
  é um grupo cíclico.
\end{Cor}

\begin{dem}
  Do teorema\xspace\ref{teogeradoasubgrupo},
  $(\gerado{a},\ast_{\gerado{a}})$ é um grupo e, pela própria definição
  de $\gerado{a}$, é cíclico.
\end{dem}

\begin{Lem}\label{lemapotordae}
  Sendo $(G,\ast)$ um grupo finito
  e $a$ um elemento de $G$, $\cardi{\gerado{a}}$ é
  o menor inteiro positivo $t$ tal que $a^{\ast t}=e$, sendo $e$ o
  elemento neutro de $(G,\ast)$.
\end{Lem}

\begin{dem}
  Do lema\xspace\ref{lemate}, temos a garantia da existência de um
  inteiro positivo $t$ tal que $a^{\ast t}=e$. Vamos, então, mostrar
  primeiramente que
  $a^{\ast \cardi{\gerado{a}}}=e$ e
  depois que se um inteiro positivo $k$ é
  estritamente menor que $\cardi{\gerado{a}}$ então $a^{\ast k}\neq e$.

  Contrapositivamente, se $a^{\ast \cardi{\gerado{a}}}\neq e$ então,
  como $e\in\gerado{a}$, existe um inteiro
  positivo $j<\cardi{\gerado{a}}$ tal que $a^{\ast j}=e$ e, portanto,
  para todo inteiro $\ell>j$, $a^{\ast \ell}=a^{\ast (\ell-j)}$, o que
  nos leva a concluir que $\cardi{\gerado{a}}=j$, o que é um absurdo.
  Notemos ainda que esse mesmo argumento nos permite verificar que não
  pode existir um inteiro estritamente menor que $\cardi{\gerado{a}}$
  tal que $a^{\ast k}\neq e$, como queríamos mostrar.
\end{dem}

\begin{Teo}\label{teoacardige}
  Em qualquer grupo finito $(G,\ast)$ com elemento neutro $e$
  vale que
  \begin{equation*}
    a^{\ast\cardi{G}} = e
  \end{equation*}
  para todo $a\in G$.
\end{Teo}

\begin{dem}
  Do teorema\xspace\ref{teolagrange} (teorema de Lagrange),
  $\cardi{\gerado{a}}\divd \cardi{G}$, já que, do
  teorema\xspace\ref{teogeradoasubgrupo}, $\gerado{a}$ é um subgrupo de
  $(G,\ast)$. Assim, sendo $q$ um inteiro tal que
  $\cardi{G}=\cardi{\gerado{a}}q$, temos, da
  propriedade\xspace\ref{proprpotmnpotmn}, que
  $a^{\ast \cardi{G}} = {\bigl(a^{\ast
  \cardi{\gerado{a}}}\bigr)}^q$. Portanto, do
  lema\xspace\ref{lemapotordae}, $a^{\ast \cardi{G}} = e^q = e$, como
  queríamos mostrar.
\end{dem}

\subsection{Subgrupos normais e grupos quocientes}

\begin{Nom}\label{nomsubgruponormal}
  Dizemos que um subgrupo $N$ de um grupo $(G,\ast)$ é um
  \conceito{subgrupo normal} de $(G,ast)$
  quando, e somente quando, para todo
  elemento $x$ de $G$,
  \begin{equation*}
    x\ast N = N\ast x\MMp
  \end{equation*}
\end{Nom}

\begin{Not}
  Sendo $N$ um subgrupo normal de $(G,\ast)$, utilizamos $\simb[grupo
  quociente de $G$ por $N$]{\tcquoc{G}{N}}$
  para denotar o conjunto quociente de $G$ por qualquer uma das relações
  $\sim$ e $\backsim$ módulo $N$.
\end{Not}

\begin{Not}\label{notmulsubcj}
  Sendo $N_1$ e $N_2$ subgrupos quaisquer de um grupo $(G,\ast)$,
  utilizamos $\simb[multiplicação dos subgrupos normais $N_1$ e
  $N_2$]{N_1\ast N_2}$ para
  denotar o conjunto
  \begin{equation*}
    N_1 = \cjpp{n\ast N_2}{n\in N_1}\MMp
  \end{equation*}
\end{Not}

\begin{Obs}
  Notemos
  que, por causa da nomenclatura\xspace\ref{nomsubgruponormal}, uma
  definição equivalente para a notação\xspace\ref{notmulsubcj} seria:
  \begin{equation*}
    N_1 = \cjpp{N_1\ast n}{n\in N_2}\MMp
  \end{equation*}
\end{Obs}

\begin{Lem}\label{lemaNbNabN}
  A operação $\ast$, como convencionamos na
  notação\xspace\ref{notmulsubcj}, é uma operação fechada sobre o
  conjunto quociente $\tcquoc{G}{N}$.
\end{Lem}

\begin{dem}
  Mostraremos simplesmente que, sendo $a$ e $b$ elementos de $G$,
  $(a\ast N)\ast(b\ast N) = (a\ast b)\ast N$. Seja $x\in (a\ast N)\ast
  (b\ast N)$. Assim, da notação\xspace\ref{notmulsubcj}, existe um
  elemento
  $a_1\in(a\ast N)$ e um elemento $b_1\in(b\ast N)$ tais que $x=a_1\ast
  b_1$. Sabemos que $a_1 = a\ast n_a$ e $b_1 = b\ast n_b$ para algum
  $n_a$ e algum $n_b$ em $N$. Portanto,
  \begin{equation*}
    x = (a\ast n_a)(b\ast n_b) = a\ast(n_a\ast b)\ast n_b\MMp
  \end{equation*}
  Mas, como $(n_a\ast b)\in (N\ast b)$, e como $(N\ast b)=(b\ast N)$, há
  algum $n\in N$ tal que $n_a\ast b = b\ast n$. Dessarte,
  \begin{equation*}
      x = a\ast(n_a\ast b)\ast n_b = a\ast(b\ast n)\ast n_b
        = (a\ast b)\ast(n\ast n_b)\MMv
  \end{equation*}
  e, já que $(n\ast n_b)\in N$, $x\in(a\ast b)\ast N$, o que nos traz
  que
  \begin{equation*}
    (a\ast N)\ast(b\ast N) \subseteq (a\ast b)\ast N\MMp
  \end{equation*}

  Mostraremos agora que $(a\ast b)\ast N\subseteq(a\ast N)\ast(b\ast
  N)$. Tomemos, para tanto, $x\in(a\ast b)\ast N$, o que garante a
  existência de um elemento $n$ de $N$ tal que $x=(a\ast b)\ast
  n$. Mas, sendo $e$ o elemento neutro de $(G,\ast)$, temos que
  $x = (a\ast e)\ast (b\ast n)$,
  e, como $e\in N$, já que $N$ é um subgrupo, podemos finalmente
  concluir que $x\in
  (a\ast N)\ast(b\ast N)$.
\end{dem}

\begin{Teo}\label{teogrupoquociente}
  Sendo $N$ um subgrupo normal de um grupo $(G,\ast)$, a operação
  $\ast$
  e o conjunto quociente $\tcquoc{G}{N}$ definem um grupo, chamado de
  \conceito{grupo quociente} de $G$ por $N$.
\end{Teo}

\begin{dem}
  Por causa do lema\xspace\ref{lemaNbNabN}, $\ast$ pode ser considerada
  uma operação sobre $\tcquoc{G}{N}$. É fácil verificar que $\ast$ é
  associativa, já que, sendo $a$, $b$ e $c$ elementos de $G$,
  \begin{equation*}
    \begin{aligned}
      \bigl((a\ast N)\ast(b\ast N)\bigr)\ast (c\ast N)
      &\subseteq (a\ast N)\ast\bigl((b\ast N)\ast(c\ast N)\bigr)
      \quad\text{e}\\
      (a\ast N)\ast\bigl((b\ast N)\ast(c\ast N)\bigr) &\subseteq
      \bigl((a\ast N)\ast(b\ast N)\bigr)\ast (c\ast N)\MMv
    \end{aligned}
  \end{equation*}
  uma vez que se $x\in\bigl((a\ast N)\ast(b\ast N)\bigr)\ast(c\ast N)$
  então há $n_a$, $n_b$
  e $n_c$ em
  $N$ tais que
  \begin{equation*}
    \begin{aligned}
      x &= \bigl((a\ast n_a)\ast (b\ast n_b)\bigr)\ast (c\ast n_c)\\
        &= (a\ast n_a)\ast \bigl((b\ast n_b)\ast (c\ast n_c)\bigr)\MMv
    \end{aligned}
  \end{equation*}
  da mesma forma que se
  $x\in(a\ast N)\ast\bigl((b\ast N)\ast(c\ast N)\bigr)$
  então há $n_a$, $n_b$
  e $n_c$ em
  $N$ tais que
  \begin{equation*}
    \begin{aligned}
      x &= (a\ast n_a)\ast \bigl((b\ast n_b)\ast (c\ast n_c)\bigr)\\
        &= \bigl((a\ast n_a)\ast (b\ast n_b)\bigr)\ast (c\ast n_c)\MMp
    \end{aligned}
  \end{equation*}
  Também a existência de um elemento neutro para $(\tcquoc{G}{N},\ast)$
  pode ser averiguada se notarmos que, do lema\xspace\ref{lemaNbNabN},
  para todo $a\in G$,
  \begin{equation*}
    (a\ast N)\ast(e\ast N) = (a\ast e)\ast N = a\ast N\MMp
  \end{equation*}
  Por último, também por causa do lema\xspace\ref{lemaNbNabN}, é verdade
  que
  \begin{equation*}
    (a\ast N)\ast(\overline{a}\ast N) =
    (a\ast \overline{a})\ast N =
    e\ast N\MMv
  \end{equation*}
  e, portanto, todo elemento de $\tcquoc{G}{N}$ possui um simétrico em
  $(\tcquoc{G}{N},\ast)$ em relação ao elemento neutro $(e\ast N)$.
\end{dem}

\section{Anéis e estruturas afins}

\subsection{Anéis}

\begin{Def}
  Sendo um conjunto $R$ munido de duas operações, $+$ e $\cdot$, dizemos
  que $(R,+,\cdot)$ é um \conceito{anel} se e só se:
  \begin{enumerate}[(i)]
    \item $(R,+)$ é um grupo abeliano;
    \item $\cdot$ é uma operação associativa;
    \item $\cdot$ é distributiva em relação a $+$.
  \end{enumerate}
\end{Def}

\begin{Not}
  Em um anel cujas operações são, na ordem da definição, denotadas por
  $+$ e $\cdot$, sendo $r$ qualquer elemento do anel, denotamos por
  \begin{enumerate}[(i)]
    \item $\simb[elemento neutro de uma operação denotada por
    $+$]{\zero}$ o elemento neutro de $+$ e
    \item $\simb[simétrico de $r$ em relação a uma operação
    denotada por $+$]{-r}$ o elemento simétrico de $r$ em relação a $+$;
  \end{enumerate}
\end{Not}

\begin{Obs}
  Num anel $(R,+,\cdot)$, sempre assumimos a precedência de $\cdot$
  sobre $+$. É também costume escrever simplesmente $\simb[$a\cdot
  b$]{ab}$ ao invés de
  $a\cdot b$, assim como $\simb[$a+(-b)$]{a-b}$ ao invés de
  $a+(-b)$.
\end{Obs}

Além das propriedades de que um anel $(R,+,\cdot)$ goza por $(R,+)$ ser
um grupo abeliano, outras demonstramos a seguir.

\begin{Propr}\label{rzeroigualazero}
  Sendo $(R,+,\cdot)$ um anel,
  \begin{equation*}
    r\cdot \zero = \zero\cdot r = \zero\MMp
  \end{equation*}
\end{Propr}

\begin{dem}
  Sabemos que
  \begin{equation*}
    \begin{aligned}
      \zero + \zero\cdot r &= \zero\cdot r\\
                   &= (\zero+\zero)r\\
                   &= \zero\cdot r + \zero\cdot r\MMv
    \end{aligned}
  \end{equation*}
  o que nos leva a concluir,
  por causa da propriedade\xspace\ref{regulargrupo}, que
  $\zero = \zero\cdot r$. Analogamente, a mesma propriedade, porque
  \begin{equation*}
    \begin{aligned}
      \zero + r\cdot \zero &= r\cdot \zero\\
                   &= r(\zero+\zero)\\
                   &= r\cdot \zero + r\cdot \zero\MMv
    \end{aligned}
  \end{equation*}
  também nos leva a concluir que $\zero = r\cdot \zero$, e, assim,
  encerramos nossa
  demonstração.
\end{dem}

\begin{Propr}
  Sendo $(R,+,\cdot)$ e sendo $a$ e $b$ elementos de $R$,
  \begin{equation*}
    a(-b) = (-a)b = -(ab)\MMp
  \end{equation*}
\end{Propr}

\begin{dem}
  Sabemos, dos axiomas da definição de anel e da
  propriedade\xspace\ref{rzeroigualazero}, que
  \begin{equation*}
    \begin{aligned}
      ab + \bigl(-(ab)\bigr) &= \zero\\
                             &= a\cdot \zero\\
                             &= a\bigl(b+(-b)\bigr)\\
                             &= ab + a(-b)\MMv
    \end{aligned}
  \end{equation*}
  o que nos leva a concluir,
  por causa da propriedade\xspace\ref{regulargrupo}, que
  $-(ab) = a(-b)$. Analogamente, a propriedade\xspace\ref{regulargrupo},
  porque
  \begin{equation*}
    \begin{aligned}
      ab + \bigl(-(ab)\bigr) &= \zero\\
                             &= \zero\cdot b\\
                             &= \bigl(a+(-a)\bigr)b\\
                             &= ab + (-a)b\MMv
    \end{aligned}
  \end{equation*}
  também nos leva a concluir que $-(ab) = (-a)b$, e, assim,
  encerramos nossa
  demonstração.
\end{dem}

\begin{Nom}
  Dizemos que um anel $(R,+,\cdot)$ é \Conceito{finito}{anel finito}
  quando
  e só quando $R$ é finito.
\end{Nom}

\begin{Not}
  Seja $(R,+,\cdot)$ um anel qualquer, seja $r$ um elemento de $R$ e
  seja $n$ um número natural.
  O $n$-ésimo \Conceito{múltiplo}{múltiplo de um elemento de um anel} de
  $r$ no anel é o elemento definido por:
  \begin{equation*}
    \simb[$n$-ésimo múltiplo de $r$]{nr} = \left\{
    \begin{aligned}
      &(n-1)r+r\MMv&\quad&\text{se $n>0$;}\\
      &\zero\MMv&\quad&\text{caso contrário.}
    \end{aligned}
    \right.
  \end{equation*}
\end{Not}

\begin{Def}
  Sendo $(R,+,\cdot)$ um anel qualquer, dizemos que um inteiro positivo
  $n$ é a \Conceito{característica}{característica de um anel} de
  $(R,+,\cdot)$ se e somente se
  \begin{equation*}
    n = \min{\cjpp{\ell\in\MMN\setminus{\cj{0}}}%
      {\ell r = \zero\text{ para todo }r\in R}}\MMp
  \end{equation*}
  Se, entretanto, $\cjpp{\ell\in\MMN\setminus{\cj{0}}}%
      {\ell r = \zero\text{ para todo }r\in R}=\emptyset$, dizemos que a
      característica do anel $(R,+,\cdot)$ é igual a $0$.
\end{Def}

\subsection{Subanéis}

\begin{Def}
  De um anel $(R,+,\cdot)$ dizemos que um subconjunto
  não-vazio $L$ de $R$ é
  um
  \conceito{subanel} e escrevemos
  \begin{equation*}
    \simb[propriedade de subanel de $L$ em relação a
    $(R,+,\cdot)$]{L\subseteq (R,+,\cdot)}
  \end{equation*}
  se e somente se:
  \begin{enumerate}[(i)]
    \item $L$ é fechado para ambas as operações $+$ e $\cdot$;
    \item $(L,+_H,\cdot_H)$ também é um anel.
  \end{enumerate}
\end{Def}

\begin{Teo}\label{teosubanel}
  Sendo $(R,+,\cdot)$ um anel e $L$ um subconjunto não vazio de $R$, $L$
  é um subanel de $(R,+,\cdot)$ se e só se $a-b$ e $ab$ pertencerem a
  $L$
  para todo $a$ e todo $b$ elementos de $L$.
\end{Teo}

\begin{dem}
  Mostremos inicialmente que se $L$ é um subanel de $(R,+,\cdot)$
  então $a-b$ e $ab$
  pertencem a $L$, sendo $a$ e $b$ elementos de L.
  Para tanto, observemos que $L$ é também um subgrupo
  de $(R,+)$, e, portanto, $a-b = a+(-b)$ é assim um elemento de
  $L$, do teorema\xspace\ref{eqsubgrupo}.
  Por sua vez, como $L$ é fechado para a operação $\cdot$, $ab\in
  L$ de igual maneira.

  Agora mostremos que se $a-b$ e $ab$ pertencem a $L$
  para todo $a$ e todo $b$ elementos de $L$
  então $L$ é um
  subanel de $(R,+,\cdot)$.
  Novamente do teorema\xspace\ref{eqsubgrupo},
  sabemos que $L$ é um subgrupo de $(R,+)$, já que $a-b\in
  L$, o que nos garante inclusive que $+_L$ se trata de uma operação
  fechada sobre $R$.
  Notemos que $\cdot_L$ também é fechada
  sobre $R$, já que, da hipótese,
  $ab\in L$ para todo $a$ e todo $b$ elementos de
  $L$. Falta-nos, assim, mostrar que $(L,+_L,\cdot_L)$ satisfaz os
  axiomas de anel. No entanto,
  a comutatividade de $+_L$ é imediata, e,
  portanto, $(L,+_L)$ é um grupo abeliano, já que, como mencionamos,
  $L$ é um subgrupo de $(R,+)$.
  Como a associatividade da operação $\cdot_L$ também é imediata,
  resta-nos
  assim apenas mostrar sua distributividade em relação a $+_L$, o que
  se verifica porque
  \begin{equation*}
    a\cdot_L(b+_Lc) = a(b+c) = ab + ac =
    a\cdot_Lb +_L a\cdot_Lc\MMp
  \end{equation*}
  Assim, concluimos nossa demonstração.
\end{dem}

\subsection{Anéis comutativos}

\begin{Def}
  Dizemos que um anel $(R,+,\cdot)$ é \Conceito{comutativo}{anel
  comutativo} se e só se $\cdot$ satisfaz a propriedade da
  comutatividade.
\end{Def}

\subsection{Anéis com unidade}

\begin{Def}
  Dizemos que um anel $(R,+,\cdot)$ é um \conceito{anel com unidade} se
  e somente se a operação $\cdot$ possui elemento neutro, chamado de
  \Conceito{unidade}{unidade de um anel} do anel.
\end{Def}

\begin{Propr}
  Um anel $(R,+,\cdot)$ pode possuir no máximo uma unidade.
\end{Propr}

\begin{dem}
  Sejam $u_1$ e $u_2$ unidades de um anel com unidade
  $(R,+,\cdot)$. Como $u_1$ é um elemento de $R$ e
  $u_2$ é um elemento neutro para $\cdot$, temos que
  \begin{equation*}
    u_1 = u_1\cdot u_2 = u_2\cdot u_1\MMp
  \end{equation*}
  Como $u_2$ é um elemento de $R$ e
  $u_1$ é um elemento neutro, temos que
  \begin{equation*}
    u_2\cdot u_1 = u_2\MMp
  \end{equation*}
  Portanto, $u_1 = u_2$.
\end{dem}

\begin{Propr}
  Em um anel com unidade $(R,+,\cdot)$, cada elemento $r$ de $R$ só pode
  possuir no máximo um simétrico em relação a $\cdot$.
\end{Propr}

\begin{dem}
  Seja $u$ a unidade de um anel com unidade $(R,+,\cdot)$ e seja $s$ um
  elemento de $R$ simetrizável em relação a $\cdot$, sendo $s_1$ e $s_2$
  elementos simétricos de $s$. Como $s_1 = s_1\cdot u$ e $u = s\cdot
  s_2$, temos que
  \begin{equation*}
    \begin{aligned}
      s_1 &= s_1\cdot(s\cdot s_2)\\
          &= (s_1\cdot s)\cdot s_2\\
          &= u\cdot s_2\\
          &= s_2\MMv
    \end{aligned}
  \end{equation*}
  como queríamos demonstrar.
\end{dem}

\begin{Not}
  Em um anel com unidade
  cujas operações são, na ordem da definição, denotadas por
  $+$ e $\cdot$, sendo $r$ um elemento
  simetrizável em relação a $\cdot$, denotamos por
  \begin{enumerate}[(i)]
    \item $\simb[elemento neutro de uma operação denotada por
    $\cdot$]{\um}$ o elemento neutro de $\cdot$ e
    \item $\simb[elemento simétrico de $r$ em relação a uma operação
    denotada por $\cdot$]r^{-1}$ o elemento simétrico de $r$ em relação
    a $\cdot$.
  \end{enumerate}
\end{Not}

\begin{Propr}\label{proprnmrsnrms}
  Sendo $(R,+,\cdot)$ um anel com unidade, sendo $r$ e $s$
  elementos de $R$ e
  sendo $n$ e $m$ números naturais,
  \begin{equation*}
    (nm)(rs) = (nr)(ms)\MMp
  \end{equation*}
\end{Propr}

\begin{dem}
  Sabemos que $(0m)(rs) = 0 = 0(ms) = (0r)(ms)$. Tomemos um natural $k$
  e, por indução em $n$,
  suponhamos que valha, para todo $\ell\in[0..k]$, que $(\ell m)(rs) =
  (\ell r)(ms)$. Queremos, então, somente mostrar que $((k+1)m)(rs) =
  ((k+1)r)(ms)$. Ora,
  \begin{equation*}
      ((k+1)m)(rs) = (km+m)(rs) \MMv
  \end{equation*}
  e, da propriedade\xspace\ref{apotm-apotn-apotmn},
  \begin{equation*}
    ((k+1)m)(rs) = (km)(rs)+(1m)(rs)\MMv
  \end{equation*}
  e, da hipótese da indução, como $1$ e $k$ são elementos de $[0..k]$,
  \begin{equation*}
    ((k+1)m)(rs) = (kr)(ms)+r(ms)
  \end{equation*}
  e, da distributividade de $\cdot$ sobre $+$,
  \begin{equation*}
      ((k+1)m)(rs) = \bigl((kr)+r\bigr)(ms)\MMv
  \end{equation*}
  e, novamente da propriedade\xspace\ref{apotm-apotn-apotmn},
  \begin{equation*}
    ((k+1)m)(rs) = \bigl((k+1)r\bigr)(ms)\MMv
  \end{equation*}
  como queríamos mostrar.
\end{dem}

\begin{Def}\label{defpotencia}
  Seja $(R,+,\cdot)$ um anel com unidade, seja $r$ um elemento de $R$ e
  seja $n$ um número natural.
  A $n$-ésima \conceito{potência} de $r$ no anel
  é o elemento definido por:
  \begin{equation*}
    \simb[$n$-ésima potência de $r$]{r^n} = \left\{
    \begin{aligned}
      &(r^{n-1})r\MMv&\quad&\text{se $n>0$;}\\
      &\um\MMv&\quad&\text{caso contrário.}
    \end{aligned}
    \right.
  \end{equation*}
\end{Def}

\begin{Obs}
  Sendo $m$ um inteiro tal que
  $|m|>1$ e definindo-se $\funcao{+}{\MMZ_m\times\MMZ_m}%
  {\MMZ_m}$ e $\funcao{\cdot}{\MMZ_m\times\MMZ_m}{\MMZ_m}$ por
  \begin{equation*}
    \begin{aligned}
      \cleq{a}{m}+\cleq{b}{m} &= \cleq{a+b}{m}\quad\text{e}\\
      \cleq{a}{m}\cleq{b}{m}  &= \cleq{ab}{m}\MMv
    \end{aligned}
  \end{equation*}
  é imediato verificar que $(\MMZ_m,+,\cdot)$ se trata de um anel com
  unidade, cuja unidade é $\cleq{1}{m}$.
\end{Obs}

\subsection{Domínios de integridade}

\begin{Def}
  Dizemos que um anel comutativo com unidade $(R,+,\cdot)$ é um
  \conceito{anel de integridade}, ou \conceito{domínio de integridade},
  se e somente se para quaisquer $a$ e $b$ elementos de $R$ vale
  a \conceito{lei do anulamento do produto}:
  se
  $ab=\zero$ então $a=\zero$ ou $b=\zero$.
\end{Def}

\begin{Teo}\label{teodiprimo}
  Sendo $m$ um inteiro tal que $|m|>1$,
  $(\MMZ_m,+,\cdot)$ é um domínio de integridade
  se e só se $m$ é um número primo.
\end{Teo}

\begin{dem}
  Suponhamos inicialmente que $(\MMZ_m,+,\cdot)$
  seja um domínio de integridade e suponhamos que $m$ não seja
  primo, mais especificamente, como $|m|>1$, composto. Assim, existem
  $a$ e $b$ em $[1..(m-1)]$ tais que $|m|=ab$ e, conseqüentemente,
  da definição de $\cdot$ sobre $\MMZ_m$,
  \begin{equation*}
    \cleq{a}{m}\cleq{b}{m} = \cleq{ab}{m}
    = \cleq{|m|}{m} = \cleq{0}{m}\MMv
  \end{equation*}
  contrariando a lei do anulamento do produto.

  Falta-nos, portanto, somente mostrar que se $m$ é primo então
  $(\MMZ_m,+,\cdot)$ é um domínio de integridade. Já sabemos que
  $(\MMZ_m,+,\cdot)$ se trata de um anel comutativo com unidade
  $\cleq{1}{m}$, pois, para todo inteiro $z$, $\cleq{z}{m}\cleq{1}{m} =
  \cleq{z\dot 1}{m} = \cleq{z}{m}$.
  Dessarte, resta-nos apenas
  mostrar que vale a lei do anulamento do produto. Sejam $a$ e $b$
  inteiros e suponhamos,
  então, que $\cleq{a}{m}\cleq{b}{m}=\cleq{0}{m}$.
  Queremos, assim, mostrar que $\cleq{a}{m}=\cleq{0}{m}$ ou
  $\cleq{b}{m}=\cleq{0}{m}$.
  Sabemos, como $\cleq{a}{m}\cleq{b}{m}=\cleq{ab}{m}=\cleq{0}{m}$,
  que $m\divd ab$. Mas,
  da propriedade\xspace\ref{primodivide}, como
  $m$ é primo, $m\divd a$ ou $m\divd b$. Finalmente,
  $\cleq{a}{m}=\cleq{0}{m}$
  ou
  $\cleq{b}{m}=\cleq{0}{m}$.
\end{dem}

\begin{Teo}\label{integridaderegular}
  Sendo $(R,+,\cdot)$ um anel comutativo com unidade, $(R,+,\cdot)$ é um
  domínio de integridade se e só se todo elemento de
  $R\setminus\cj{\zero}$
  é regular para $\cdot$.
\end{Teo}

\begin{dem}
  Suponhamos que
  $(R,+,\cdot)$ seja um domínio de integridade. Sejam $x$ e $y$
  elementos quaisquer de $R$ e seja $a$ um elemento de $R$ diferente de
  $\zero$. Suponhamos que $ax = ay$.
  Assim,
  \begin{equation*}
    ax - ay = \zero\MMv
  \end{equation*}
  e, portanto,
  \begin{equation*}
    a(x-y) = \zero\MMp
  \end{equation*}
  Como $a\neq\zero$ e $(R,+,\cdot)$ se trata de um domínio
  de integridade,
  temos então que $x-y=0$ e, conseqüentemente, que $x=y$.

  Agora, para completarmos a prova, suponhamos que todo elemento de
  $R\setminus\cj{\zero}$
  seja regular para $\cdot$. Entretanto, suponhamos
  também que existam $a$ e $b$ elementos de $R$
  diferentes de $\zero$ tais que $ab=\zero$. Temos então que
  \begin{equation*}
    ab = \zero = a\cdot \zero\MMpv
  \end{equation*}
  mas, como $a$ é regular para $\cdot$, concluímos que $b=\zero$, o que
  é um absurdo. Assim, vale a lei do anulamento do produto.
\end{dem}

\begin{Teo}\label{teocaracanel}
  Sendo $(R,+,\cdot)$ um anel (não necessariamente comutativo)
  com unidade no qual vale a lei do
  anulamento do produto, se $\cardi{R}>1$
  e se $(R,+,\cdot)$ tem característica positiva
  então a característica de $(R,+,\cdot)$ é um número
  primo.
\end{Teo}

\begin{dem}
  Seja $n$ a característica de $(R,+,\cdot)$
  Como $\cardi{R}>1$, podemos tomar um $r\in R\setminus\cj{\zero}$ e,
  como $nr=\zero$, concluir que $n\geq 2$. Suponhamos que $n$ não seja
  um número primo. Então, como $n$ é positivo no mínimo $2$,
  existem $k$ e $m$ em
  $[2..(n-1)]$ tais que $n=km$. Assim,
  \begin{equation*}
    n\um = (km)\um = (km)(\um\cdot\um)\MMv
  \end{equation*}
  e, da propriedade\xspace\ref{proprnmrsnrms},
  \begin{equation*}
    n\um = (k\cdot \um)(m\cdot \um) = \zero\MMv
  \end{equation*}
  e, como vale a lei do anulamento do produto, $k\cdot \um=\zero$ ou
  $m\cdot \um=\zero$. Portanto, para todo $r\in R$, pelas
  propriedades\xspace\ref{proprnmrsnrms} e\xspace\ref{rzeroigualazero},
  \begin{equation*}
    kr = (k\cdot 1)(\um\cdot r) = (k\cdot\um)r = \zero\cdot r = \zero
    \qquad\text{ou}\qquad
    mr = (m\cdot 1)(\um\cdot r) = (m\cdot\um)r = \zero\cdot r = \zero
    \MMv
  \end{equation*}
  o que é um absurdo, já que $n$ é o menor inteiro positivo $\ell$
  tal que $\ell r=\zero$.
\end{dem}

\begin{Nom}
  Dizemos que um domínio de integridade
  $(R,+,\cdot)$ é \Conceito{finito}{domínio de integridade finito}
  quando e só quando $R$ é finito. Ademais, dizemos que $\cardi{R}$ é a
  \Conceito{ordem}{ordem de um domínio de integridade finito} de
  $(R,+,\cdot)$.
\end{Nom}

\subsection{Corpos}

\begin{Def}
  Sendo $(F,+,\cdot)$ um anel comutativo com unidade, $(F,+,\cdot)$ é um
  \conceito{corpo}\footnote{Em inglês,
  \textit{field}\index{\textit{field}}.}
  se e só se todo elemento de $F\setminus\cj{\zero}$ é
  simetrizável em relação e $\cdot$.
\end{Def}

\begin{Teo}\label{teocorpodi}
  Todo corpo é um domínio de integridade.
\end{Teo}

\begin{dem}
  Seja $(F,+,\cdot)$ um corpo. Tomemos $a$ e $b$ elementos quaisquer de
  $F$ e suponhamos que $ab = \zero$. Se, no entanto, supuséssemos que
  $a\neq\zero$ e $b\neq\zero$, teríamos, por causa da existência de
  $a^{-1}$ e de $b^{-1}$ garantida pela definição de corpo, que
  \begin{equation*}
    \begin{aligned}
      \zero &= a^{-1}\cdot\zero\\
            &= a^{-1}(ab) \\
            &= (a^{-1}a)b \\
            &= b\MMv
    \end{aligned}
  \end{equation*}
  o que seria um absurdo. Assim, vale a lei do anulamento do produto, e,
  dessarte, concluimos que $(F,+,\cdot)$ também é um domínio de
  integridade.
\end{dem}

\begin{Teo}\label{teodifinitocorpo}
  Todo domínio de integridade finito é um corpo.
\end{Teo}

\begin{dem}
  Seja $(R,+,\cdot)$ um domínio de integridade finito.
  Como $R\setminus\cj{\zero}\neq\emptyset$, já que ao menos $\um\in
  R\setminus\cj{\zero}$,
  tomemos um elemento $a$ de $R$ diferente
  de $\zero$. Vamos encontrar um elemento simétrico para $a$ em relação
  a $\cdot$. Definamos a função $\funcao{f_a}{R}{R}$ por:
  \begin{equation*}
    f_a(r) = ar\MMp
  \end{equation*}
  Sejam $x$ e $y$ elementos de $R$. Vamos mostrar que se $x\neq y$ então
  $f_a(x)\neq f_a(y)$ através da forma contrapositiva. Suponhamos que
  $f_a(x)=f_a(y)$. Portanto, $ax = ay$,
  e, assim, do teorema\xspace\ref{integridaderegular}, $x=y$. Mostrado
  que $f_a$ se trata de uma injeção,
  notemos que se trata também de uma bijeção, já que possui domínio e
  contradomínio finitos e idênticos. Assim, $\um\in f_a(R)$, e,
  dessarte, existe um $r_1\in R$ tal que $f_a(r_1)=\um$. Notemos,
  ademais,
  que
  \begin{equation*}
    f_a(r_1) = ar_1 = \um\MMp
  \end{equation*}
  Logo, $r_1$ é simétrico de $a$ em relação a $\cdot$, como
  procurávamos.
\end{dem}

\subsection{Corpos finitos}

\begin{Nom}
  Dizemos que um corpo $(F,+,\cdot)$ é \Conceito{finito}{corpo finito}
  quando e só quando $F$ é finito. Ademais, dizemos que $\cardi{F}$ é a
  \Conceito{ordem}{ordem de um corpo finito} de $(F,+,\cdot)$.
\end{Nom}

\begin{Teo}\label{teoacardifa}
  Sendo $(F,+,\cdot)$ um corpo finito, vale para todo $a$ em $F$ que
  \begin{equation*}
    a^{\cardi{F}} = a\MMp
  \end{equation*}
\end{Teo}

\begin{dem}
  Seja $a$ um elemento de $F$. Se $a=\zero$ então a igualdade se
  verifica trivialmente. Se, por outro lado, $a\neq\zero$ então $a\in
  F\setminus\cj{0}$ e, como $(F\setminus\cj{0},\cdot)$ é um
  grupo finito,
  do teorema\xspace\ref{lemate}, $a^{\cardi{F}-1}=\um$, e,
  portanto,
  \begin{equation*}
    a^{\cardi{F}} = a\cdot a^{\cardi{F}-1} = a\cdot \um = a\MMv
  \end{equation*}
  como queríamos mostrar.
\end{dem}

\begin{Teo}\label{teocaracacorpo}
  Todo corpo finito possui característica prima.
\end{Teo}

\begin{dem}
  Seja $(F,+,\cdot)$ um corpo finito.
  Vamos primeiramente mostrar que a característica de $(F,+,\cdot)$ é
  positiva. Como $(F,+)$ é um grupo finito, então, do
  lema\xspace\ref{lemate},
  podemos tomar $t$ o menor inteiro
  positivo tal que $t\cdot\um=\zero$. Assim, para todo $r\in F$ vale,
  já que $(F\setminus{\cj{\zero}},\cdot)$ é um grupo, por causa das
  propriedades\xspace\ref{abpotn-apotnbpotn}
  e\xspace\ref{rzeroigualazero},
  \begin{equation*}
    tr = t(\um\cdot r) = (t\cdot\um)\cdot(tr) = \zero\cdot(tr) =
    \zero\MMp
  \end{equation*}
  Para concluirmos que $t$ se trata da característica de $(F,+,\cdot)$,
  basta mostrarmos que não existe um inteiro positivo $k<t$ tal que
  $kr = \zero$ para todo $r\in F$. Supondo, no entanto, a existência
  desse $k$, temos que $k\cdot\um=\zero$, o que contraria a escolha de
  $t$.

  Mostrado que a característica de $(F,+,\cdot)$ é positiva, a
  demonstração se conclui por causa do teorema\xspace\ref{teocaracanel}
  na medida em que $(F,+,\cdot)$ se trata de um anel com unidade e na
  medida em que, do teorema\xspace\ref{teocorpodi}, vale em
  $(F,+,\cdot)$ a lei do anulamento do produto.
\end{dem}

\begin{Obs}\label{obsabpnapnbpn}
  Um resultado muito conhecido que não demonstraremos mas usaremos no
  presente trabalho é que se $a$ e $b$ são elementos de um corpo finito
  com característica $p$ então
  \begin{equation*}
    (a+b)^{p^n} = a^{p^n} + b^{p^n}\MMv
  \end{equation*}
  para todo natural $n$.
\end{Obs}

\subsection{Subcorpos}

\begin{Def}
  Sendo $(F,+,\cdot)$ um corpo, dizemos que um subconjunto não vazio
  $K$ de $F$ é
  um \conceito{subcorpo} e escrevemos
  \begin{equation*}
    \simb[propriedade de subcorpo de $K$ em relação a
    $(F,+,\cdot)$]{K\subseteq (F,+,\cdot)}
  \end{equation*}
  se e somente se:
  \begin{enumerate}[(i)]
    \item $K$ é fechado para ambas as operações $+$ e $\cdot$;
    \item $(K,+_K,\cdot_K)$ também é um corpo.
  \end{enumerate}
\end{Def}

\begin{Teo}\label{teosubcorpo}
  Sendo $(F,+,\cdot)$ um corpo, um subconjunto $K$ de $F$ é um subcorpo
  se e só se:
  \begin{enumerate}[({\ref{teosubcorpo}}.i)]
    \item $\zero\in K$ e $\um\in K$;
    \item se $x\in K$ e $y\in K$ então $x-y\in K$;
    \item se $x\in K$ e $y\in K\setminus\cj{\zero}$ então $xy^{-1}\in
      K$.
  \end{enumerate}
\end{Teo}

\begin{dem}
  Suponhamos
  que $K$ seja um subcorpo. Assim, como $K$ é um subgrupo abeliano
  dos grupos
  abelianos $(F,+)$ e $(F\setminus\cj{\zero},\cdot)$, temos que
  $\zero\in K$ e $\um\in K$. Sejam, então, $x$ e $y$ elementos de $K$.
  Novamente, como $K\subseteq(F,+)$, $x-y\in K$. Ainda, se $y\neq\zero$,
  como $K\subseteq(F\setminus\cj{\zero},\cdot)$, $xy^{-1}\in K$.

  Agora, suponhamos que:
  \begin{enumerate}[({\ref{teosubcorpo}}.i)]
    \item $\zero\in K$ e $\um\in K$;
    \item se $x\in K$ e $y\in K$ então $x-y\in K$;
    \item se $x\in K$ e $y\in K\setminus\cj{\zero}$ então $xy^{-1}\in
      K$.
  \end{enumerate}
  Dessarte, podemos concluir, pelo teorema\xspace\ref{eqsubgrupo},
  que $K$ se trata de
  um subgrupo abeliano tanto de
  $(F,+)$ quando de $(F,\cdot)$. Logo, $K$ é um subcorpo de
  $(F,+,\cdot)$.
\end{dem}

\begin{Nom}
  \begin{enumerate}[(i)]
    \item  Um subcorpo $K$ de um corpo $(F,+,\cdot)$ é um
      \conceito{subcorpo
      próprio} de $(F,+,\cdot)$ se e só se $K\neq F$.
    \item  Um corpo $(F,+,\cdot)$ é dito um \conceito{corpo primo} se e
      só se não
      possui subcorpos próprios.
    \item Um subcorpo $K$ de um corpo $(F,+,\cdot)$ é um
    \conceito{subcorpo primo} de $(F,+,\cdot)$ se e só se
    $(K,+_K,\cdot_K)$ é um corpo primo.
  \end{enumerate}
\end{Nom}

\begin{Def}
  Dizemos que dois corpos $(F_1,+_1,\cdot_1)$ e $(F_2,+_2,\cdot_2)$ são
  \Conceito{isomorfos}{corpos isomorfos} e escrevemos
  $(F_1,+_1,\cdot_1)\simeq (F_2,+_2,\cdot_2)$ se e só se existe uma
  bijeção
  $\funcao{f}{F_1}{F_2}$ tal que, para todo $a$ e todo $b$ em $F_1$,
  \begin{equation*}
    f(a+_1b) = f(a)+_2f(b)\MMv\qquad\text{e}\qquad
    f(a\cdot_1b) = f(a)\cdot_2f(b)\MMp
  \end{equation*}
\end{Def}

\subsection{Ideais}

\begin{Def}\label{defideal}
  Sendo $(R,+,\cdot)$ um anel,
  dizemos que um subconjunto $J$ de $R$ é um \conceito{ideal}
  sobre $(R,+,\cdot)$ se e somente se
  \begin{enumerate}[(i)]
    \item $J$ é um subanel de $(R,+,\cdot)$ e
    \item para qualquer $a\in J$ e qualquer $r\in R$, $ar\in J$ e $ra\in
    J$.
  \end{enumerate}
\end{Def}

\begin{Propr}
  Sendo $J$ um ideal sobre um anel $(R,+,\cdot)$ com unidade,
  se algum elemento
    simetrizável em relação a $\cdot$ pertence a $J$ então $J=A$.
\end{Propr}

\begin{dem}
  Suponhamos que
  algum elemento simetrizável em relação a $\cdot$ pertença a $J$.
  Já sabemos
  que $J\subseteq R$, restando-nos apenas mostrar que $R\subseteq
  J$. Seja $r\in R$ e seja $x$, cuja existência é garantida pela
  hipótese,
  um elemento de $J$ simetrizável em
  relação a $\cdot$. Sabemos que
  \begin{equation*}
    r = r\cdot \um = r(u^{-1}u) = (ru^{-1})u\MMp
  \end{equation*}
  Mas, da definição de ideal, como $(ru^{-1})\in R$ e $u\in J$,
  $(ru^{-1})u\in J$, e, portanto, $r\in J$, como queríamos mostrar.
\end{dem}

\begin{Not}\label{notgerado}
  Sendo $(R,+,\cdot)$ um anel comutativo
  e $S$ um subconjunto finito de
  $R$, utilizamos $\simb[ideal gerado por $S$]{\gerado{S}}$ para denotar
  o conjunto
  \begin{equation*}
    \gerado{S} = \biggl\{{\sum_{x\in S}xf(x)
      \text{ }\biggr|\text{ }
      f\in R^S}\biggr\}
    \MMp
  \end{equation*}
  Quando $S$ é um conjunto unitário, composto apenas por um elemento
  $s$, escrevemos $\simb[ideal gerado por $\cj{s}$]{\gerado{s}}$ com o
  mesmo significado de $\gerado{S}$.
\end{Not}

\begin{Obs}\label{obsgeradounit}
  Note-se que, na particularidade da notação\xspace\ref{notgerado} para
  conjuntos unitários,
  \begin{equation*}
    \gerado{s} = \cjbar{sr}{r\in R}\MMp
  \end{equation*}
\end{Obs}

\begin{Teo}
  Sendo $(R,+,\cdot)$ um anel comutativo
  e $S$ um subconjunto finito de
  $R$, $\gerado{S}$ é um ideal sobre $(R,+,\cdot)$.
\end{Teo}

\begin{dem}
  Notemos primeiramente
  que $\gerado{S}$ não é vazio, pois pelo menos
  \begin{equation*}
    \zero = \sum_{x\in S}x\cdot \zero(x)
  \end{equation*}
  pertence a $\gerado{S}$, sendo $\funcao{\zero}{S}{R}$ a função tal que
  $\zero(x)=\zero$ para todo $x\in S$.

  Mostremos agora que $\gerado{S}$ se trata de
  um subanel de $(R,+,\cdot)$.
  Sejam $f_1$ e $f_2$ elementos de
  $A^S$.  Sabemos que
  \begin{equation*}
      \sum_{x\in S}xf_1(x) - \sum_{x\in S}xf_2(x)
        = \sum_{x\in S}x\bigl(f_1(x)-f_2(x)\bigr) \MMv
  \end{equation*}
  e, como, para todo $x\in S$, $\bigl(f_1(x)-f_2(x)\bigr)\in R$, por $R$
  ser um anel, é um elemento de $R^S$
  a função $\funcao{(f_1-f_2)}{S}{R}$ definida por:
  \begin{equation*}
    (f_1-f_2)(x) = f_1(x) - f_2(x)\MMp
  \end{equation*}
  Dessarte,
  \begin{equation*}
    \sum_{x\in S}xf_1(x) - \sum_{x\in S}xf_2(x)\in \gerado{S}\MMv
  \end{equation*}
  quaisquer que sejam $f_1$ e $f_2$ elementos de $R^S$. Como também é
  verdade que
  \begin{equation*}
    \biggl(\sum_{x\in S}xf_1(x)\biggr)\biggl(\sum_{x\in S}xf_2(x)\biggr)
    = \sum_{x\in S}x
    \Biggl(f_1(x)\biggl(\sum_{x\in S}xf_2(x)\biggr)\Biggr)
    \in \gerado{S}\MMv
  \end{equation*}
  já que
  \begin{equation*}
    f_1(x)\biggl(\sum_{x\in S}xf_2(x)\biggr)\in R
  \end{equation*}
  para todo $x\in S$, temos, do teorema\xspace\ref{teosubanel},
  que $\gerado{S}$ é um subanel de $(R,+,\cdot)$.

  Resta-nos ainda mostrar apenas que $xr\in \gerado{S}$
  para todo $x\in \gerado{S}$
  e $r\in R$, já que estamos trabalhando com um anel
  comutativo.
  Seja $f$ uma função de $S$ em $R$ e
  seja $r$ um elemento de $R$. Assim, verificamos que
  \begin{equation*}
    \biggl(\sum_{x\in S}xf(x)\biggr)r =
    \sum_{x\in S}x\bigl(f(x)r\bigr)
    \in\gerado{S}\MMv
  \end{equation*}
  uma vez que $f(x)r\in R$ para todo $x\in S$. Conseqüentemente,
  $\gerado{S}$ é um ideal sobre $(R,+,\cdot)$, como queríamos
  demonstrar.
\end{dem}

\begin{Nom}
  Sendo $(R,+,\cdot)$ um anel comutativo
  e $S$ um subconjunto finito de
  $R$, dizemos que $\gerado{S}$ é o \conceito{ideal gerado} por
  $S$. Ainda, se $\cardi{S}=1$ então dizemos que $\gerado{S}$ é um
  \conceito{ideal principal}. Se todos os ideais sobre $(R,+,\cdot)$
  são principais então dizemos que $(R,+,\cdot)$ é um
  \conceito{anel principal}.
\end{Nom}

\subsection{Anéis quocientes}

\begin{Obs}
  Sendo $J$ um ideal sobre um anel comutativo $(R,+,\cdot)$, $J$ é um
  subgrupo normal de $(R,+)$, já que $(R,+)$ é abeliano e $J$, um
  subanel de $(R,+,\cdot)$. Portanto, $(\tcquoc{R}{J},+)$ se trata de um
  grupo quociente.
\end{Obs}

\begin{Lem}\label{lema1b1Ja2b2J}
  Sendo $J$ um ideal sobre um anel comutativo $(R,+,\cdot)$, $a$ e $b$
  elementos quaisquer de $R$, $a_1$ e $a_2$ elementos
  quaisquer de $a+J$ e $b_1$ e $b_2$ elementos quaisquer de $b+J$,
  \begin{equation*}
    (a_1b_1) + J = (a_2b_2) + J
  \end{equation*}
\end{Lem}

\begin{dem}
  Como $a_1$ e $a_2$ pertencem a $a+J$,  temos da
  definição\xspace\ref{defclasselateral} que
  $-a_1+a\in J$, $-a_2+a\in J$ e, portanto,
  como $J$ é um subgrupo de $(R,+)$,
  \begin{equation*}
    -(-a_1+a)+(-a_2+a) = a_1 - a_2\in J\MMp
  \end{equation*}
  Analogamente, também temos que $b_1-b_2\in J$. Da
  definição\xspace\ref{defideal}, como $b_1$ e $a_2$ também são
  elementos de $R$, percebemos que
  \begin{equation*}
    b_1(a_1-a_2)\in J\quad\text{e}\quad a_2(b_1-b_2)\in J\MMp
  \end{equation*}
  Conseqüentemente, como $J$ é um ideal sobre $(R,+,\cdot)$,
  \begin{equation*}
    b_1(a_1-a_2) + a_2(b_1-b_2)\in J\MMv
  \end{equation*}
  e, assim,
  \begin{equation*}
    a_1b_1 - a_2b_1 + a_2b_1 - a_2b_2 = a_1b_1 - a_2b_2\in J\MMp
  \end{equation*}
  Logo, como $J$ é um subgrupo normal,
  \begin{equation*}
    \congmodright{a_1b_1}{a_2b_2}{J}\MMp
  \end{equation*}
  Dessarte, se $x\in (a_1b_1)+J$ então
  \begin{equation*}
    \begin{aligned}
      \congmodrighta{x}{a_1b_1}{J}\qquad\text{e, portanto,}\\
      \congmodrighta{x}{a_2b_2}{J}\MMv
    \end{aligned}
  \end{equation*}
  o que nos traz que $x\in (a_2b_2)+J$ e, por conseqüência, que
  \begin{equation*}
    (a_1b_1)+J\subseteq(a_2b_2)+J\MMp
  \end{equation*}
  Por outro lado, se $x\in (a_2b_2)+J$ então
  \begin{equation*}
    \begin{aligned}
      \congmodrighta{x}{a_2b_2}{J}\qquad\text{e, portanto,}\\
      \congmodrighta{x}{a_1b_1}{J}\MMv
    \end{aligned}
  \end{equation*}
  o que nos traz que $x\in (a_1b_1)+J$ e, por conseqüência, que
  \begin{equation*}
    (a_2b_2)+J\subseteq(a_1b_1)+J\MMv
  \end{equation*}
  o que completa nossa demonstração.
\end{dem}

\begin{Obs}
  O lema\xspace\ref{lema1b1Ja2b2J} nos traz que $ab+J$ continua o mesmo
  conjunto não importando quais representantes sejam escolhidos das
  classes $a+J$ e $b+J$ para substituírem respectivamente $a$ e $b$, o
  que garante que a operação introduzida pela
  notação\xspace\ref{notmultideal} seja bem definida.
\end{Obs}

\begin{Not}\label{notmultideal}
  Sendo $J$ um ideal sobre um anel comutativo $(R,+,\cdot)$, $a$ e $b$
  elementos quaisquer de $R$, utilizamos $(a+J)(b+J)$ para denotar o
  conjunto
  \begin{equation*}
    (a+J)(b+J) = (ab)+J\MMp
  \end{equation*}
\end{Not}

\begin{Obs}
  A operação $\cdot$, como convencionada na
  notação\xspace\ref{notmultideal}, é uma operação fechada sobre o
  conjunto quociente $\simb[anel de classes de resíduos de $R$ por
  $J$]{\tcquoc{R}{J}}$.
\end{Obs}

\begin{Teo}
  Sendo $J$ um ideal  sobre um anel comutativo $(R,+,\cdot)$, as
  operações $+$ e $\cdot$, nessa ordem,
  e o conjunto quociente $\tcquoc{R}{J}$ definem
  um anel, chamado de \conceito{anel quociente}
  (ou \conceito{anel de classes de resíduos}, ou \conceito{anel de
  classes residuais})
  de $R$ por $J$.
\end{Teo}

\begin{dem}
  Do teorema\xspace\ref{teogrupoquociente}, $(\tcquoc{R}{J},+)$ é o
  grupo quociente de $R$ por $J$, possuindo $+$ sobre $\tcquoc{R}{J}$
  também a
  propriedade da comutatividade, já que
  \begin{equation*}
    (x+J)+(y+J) = (x+y)+J = (y+x)+J = (y+J)+(x+J)
  \end{equation*}
  para todo $x$ e todo $y$ em $R$. Tomemos agora $a$, $b$ e $c$
  elementos de $R$. É fácil também verificar que $\cdot$
  sobre $\tcquoc{R}{J}$ também goza da associatividade, pois
  \begin{equation*}
    \begin{aligned}
      \bigl((a+J)(b+J)\bigr)(c+J) &= \bigl((ab)+J\bigr)(c+J) \\
      &= \bigl((ab)c+J\bigr) \\ &= \bigl(a(bc)+J\bigr) \\
      &= (a+J)\bigl((bc)+J\bigr) \\ &= (a+J)\bigl((b+J)(c+J)\bigr)\MMp
    \end{aligned}
  \end{equation*}
  Finalmente, notemos que vale também a distributividade de $\cdot$
  sobre $+$:
  \begin{equation*}
    \begin{aligned}
      (a+J)\bigl((b+J)+(c+J)\bigr)
        &= (a+J)\bigl((b+c)+J\bigr) \\
        &= \bigl(a(b+c)+J\bigr) \\
        &= \bigl((ab+ac)+J\bigr) \\
        &= \bigl((ab)+J\bigr) + \bigl((ac)+J\bigr) \\
        &= \bigl((a+J)(b+J)\bigr) + \bigl((a+J)(c+J)\bigr)\MMp
    \end{aligned}
  \end{equation*}
  Por tudo isso, concluimos que $(\tcquoc{R}{J},+,\cdot)$ se trata de um
  anel.
\end{dem}

\begin{Obs}\label{obszp}
  Note-se que o conjunto dos inteiros com as operações usuais de adição
  e multiplicação constituem um anel comutativo. Note-se ainda,
  da observação\xspace\ref{obsgeradounit} e da
  notação\xspace\ref{notcleqint}, que
  \begin{equation*}
    \gerado{p} = \cjpp{pz}{z\in\MMZ} = \cleq{0}{p}
  \end{equation*}
  e que
  \begin{equation*}
    \begin{aligned}
      \cquoc{\MMZ}{\gerado{p}}
      &= \bigcup_{z\in\MMZ}
        \cleq{z}{\text{$\sim$ módulo $\gerado{p}$}} \\
      &= \bigcup_{z\in\MMZ} \cjpp{x\in
        \MMZ}{\congmodright{x}{z}{\gerado{p}}} \\
      &= \bigcup_{z\in\MMZ} \cjpp{x\in\MMZ}{\congmod{x}{z}{p}} \\
      &= \bigcup_{j\in[0..(p-1)]} \cleq{j}{p}\\
      &= \MMZ_p
    \end{aligned}
  \end{equation*}
\end{Obs}

\begin{Teo}\label{teozpcorpo}
  Sendo $p$ um número primo, o anel quociente
  $(\tcquoc{\MMZ}{\gerado{p}},+,\cdot)$ é um corpo.
\end{Teo}

\begin{dem}
  Do teorema\xspace\ref{teodiprimo} e da observação\xspace\ref{obszp},
  sabemos que
  $(\tcquoc{\MMZ}{\gerado{p}},+,\cdot)$ é um domínio de
  integridade. Como
  \begin{equation*}
    \Bigl|\cquoc{\MMZ}{\gerado{p}}\Bigr| = p\MMv
  \end{equation*}
  $(\tcquoc{\MMZ}{\gerado{p}},+,\cdot)$ se trata de um domínio de
  integridade finito e, portanto, do
  teorema\xspace\ref{teodifinitocorpo}, de um corpo.
\end{dem}

\begin{Cor}
  Para um número primo $p$, seja $\simb[corpo de Galois de ordem
  $p$]{\galois{p}}$ o conjunto $[0..(p-1)]$ e
  seja $\funcao{\phi}{\tcquoc{\MMZ}{\gerado{p}}}{\galois{p}}$ a
  bijeção definida por
  \begin{equation*}
    \phi\bigl(\cleq{z}{p}\bigr)=z\MMp
  \end{equation*}
  Sejam também as operações $+$ e $\cdot$ sobre $\galois{p}$ definidas
  por:
  \begin{equation*}
    \begin{aligned}
      z_1 + z_2 &= \phi\bigl(\cleq{z_1}{p}+\cleq{z_2}{p}\bigr)\MMpv\\
      z_1\cdot z_2 &=
      \phi\bigl(\cleq{z_1}{p}\cdot\cleq{z_2}{p}\bigr)\MMp
    \end{aligned}
  \end{equation*}
  Então, a estrutura formada por $\galois{p}$ e pelas operações
  definidas acima é um corpo finito, chamado de \Conceito{corpo de
  Galois}{corpo de Galois} de \Conceito{ordem}{ordem de um corpo de
  Galois}
  $p$.
\end{Cor}

\begin{dem}
  A presente demonstração segue imediatamente do
  teorema\xspace\ref{teozpcorpo}.
\end{dem}

\begin{Propr}
  Sendo $p$ um primo positivo, todo corpo de Galois de ordem $p$ possui
  característica $p$.
\end{Propr}

\begin{dem}
  Do teorema\xspace\ref{teocaracacorpo}, segue que
  $(\galois{p},+,\cdot)$ possui
  característica prima $q$. Assim, $q\cdot 1 = 0$,
  e, portanto,
  \begin{equation*}
    \cleq{q}{p} = q\cdot\cleq{1}{p} = \cleq{0}{p}
  \end{equation*}
  no corpo
  $(\tcquoc{\MMZ}{\gerado{p}},+,\cdot)$, e, conseqüentemente,
  \begin{equation*}
    \congmod{q}{0}{p}\MMv
  \end{equation*}
  o que nos leva a concluir que $p\divd q$. Mas, como $q$ é primo e
  $|p|\neq 1$, $p=q$, como queríamos mostrar.
\end{dem}

\begin{Lem}\label{lemtcsg}
  Se $(F,+,\cdot)$ é um corpo finito de característica $q$
  então o conjunto
  \begin{equation*}
    A = \cjpp{n(\um)}{n\in\galois{q}}
  \end{equation*}
  é um subcorpo de $(F,+,\cdot)$.
\end{Lem}

\begin{dem}
  É claro que $0(\um)=\zero$ e $1(\um)=\um$ pertencem a $A$, já que
  $q\geq 2$. Temos, para todo $n\in\galois{q}$, que
  $(nq)(\um)=n(q(\um))=n(\zero)=\zero$. Assim, sendo $k\in\galois{q}$,
  é evidente que $-(k(\um))\in A$, já que
  $-(k(\um))=(q-k)(\um)$. Assim, para todo natural $\ell$, temos que
  $\ell(\um)-(k(\um))\in A$. Por último, admitamos que
  $k(\um)\neq\zero$. Como $k\in\galois{q}\setminus\cj{0}$ e
  $(\galois{q},+,\cdot)$ é um
  corpo, então
  existe um $k^{-1}\in\galois{q}$.
  Assim, $(k^{-1})(\um)$ é o inverso de $k(\um)$ em relação à
  multiplicação sobre $F$ e, portanto, para todo natural $\ell$ vale que
  $\ell(\um)(k(\um))^{-1}\in A$, e, conseqüentemente, do
  teorema\xspace\ref{teosubcorpo}, concluímos que $A$ é um subcorpo de
  $(F,+,\cdot)$.
\end{dem}

\begin{Teo}\label{teocorposimeqgalois}
  Se $(F,+,\cdot)$ é um corpo primo finito de característica $p$
  então
  $(F,+,\cdot)\simeq (\galois{p},+,\cdot)$.
\end{Teo}

\begin{dem}
  Seja a função $\funcao{f}{\galois{p}}{F}$ definida por:
  \begin{equation*}
    f(k) = k(\um)\MMp
  \end{equation*}
  Vamos mostrar que:
  \begin{enumerate}[({\ref{teocorposimeqgalois}}.i)]
    \item\label{tcsg1} $f$ é uma injeção;
    \item\label{tcsg2} $f$ é uma sobrejeção;
    \item\label{tcsg3} $f$ preserva a adição em $\galois{p}$;
    \item\label{tcsg4} $f$ preserva a multiplicação em $\galois{p}$.
  \end{enumerate}

  Mostraremos (\ref{tcsg1}) da forma contrapositiva. Tomemos $j$ e $k$
  elementos de $\galois{p}$ tais que $f(j) = f(k)$. Assim,
  $j(\um) = k(\um)$, e, da propriedade\xspace\ref{regulargrupo},
  como $(F,+)$ é um grupo, $j(\um)-k(\um)=\zero$, e, da
  propriedade\xspace\ref{apotm-apotn-apotmn}, $(j-k)(\um) = \zero$. Como
  $p$ é a característica de $(F,+,\cdot)$, $p$ é o menor inteiro
  positivo tal que $p(\um) = \zero$. Dessarte, como $j-k<p$, $j-k=0$ e,
  conseqüentemente, $j=k$, como queríamos mostrar.

  Agora, supunhamos que $f$ não seja sobrejetiva. Assim, existe um $r$
  em $F$ para o qual não há um $j\in\galois{p}$ tal que $f(j)=r$. Assim,
  o conjunto $f(\galois{p})$, do lema\xspace{lemtcsg}, é um subcorpo
  próprio de $(F,+,\cdot)$, o que é um absurdo, já que $(F,+,\cdot)$ é
  primo.

  Por fim, é imediato verificar (\ref{tcsg3}) e (\ref{tcsg4}) e concluir
  a demonstração.
\end{dem}

\section{Espaços vetoriais}

\subsection{Conceituação}

\begin{Def}
  Sendo $V$ um conjunto qualquer, $(F,+,\cdot)$ um corpo qualquer e
  $\funcao{\bfmais}{V\times V}{V}$ e $\funcao{\bfvezes}{F\times V}{V}$
  funções, dizemos que $(V,\bfmais,\bfvezes)$ é um
  \conceito{espaço vetorial} sobre $(F,+,\cdot)$ se e somente se:
  \begin{enumerate}[(i)]
    \item $(V,\bfmais)$ é um grupo abeliano;
    \item $\um\bfvezes\vetor{u}=\vetor{u}$ para todo
      $\vetor{u}\in V$;
    \item $a\bfvezes(b\bfvezes \vetor{u})
      = (a\cdot b)\bfvezes \vetor{u}$ para todo
    $\vetor{u}$ em $V$ e todo $a$ e todo $b$ em $F$;
    \item $a\bfvezes(\vetor{u}\bfmais \vetor{v})
      = (a\bfvezes \vetor{u})\bfmais(a\bfvezes \vetor{v})$
    para todo $\vetor{u}$ e todo $\vetor{v}$ em $V$ e todo $a$ em $F$;
    \item $(a+b)\bfvezes \vetor{u} =
      (a\bfvezes \vetor{u})\bfmais(b\bfvezes \vetor{u})$ para
    todo
    $\vetor{u}$ em $V$ e todo $a$ e todo $b$ em $F$.
  \end{enumerate}
\end{Def}

\begin{Nom}
  Em um espaço vetorial $(V,\bfmais,\bfvezes)$ sobre um corpo
  $(F,+,\cdot)$,
  costumamos chamar:
  \begin{enumerate}[(i)]
  \item os elementos de $V$ de \Conceito{vetores}{vetor};
  \item os elementos de $F$ de \Conceito{escalares}{escalar};
  \item a operação $\bfmais$ de \conceito{adição de vetores}, ou
  \conceito{adição vetorial};
  \item a operação $\bfvezes$ de \conceito{multiplicação por escalar}.
  \end{enumerate}
\end{Nom}

A seguir introduzimos uma notação similar àquela que introduzimos para
anéis.

\begin{Not}
  Em um espaço vetorial $(V,\bfmais,\bfvezes)$ sobre um corpo
  $(F,+,\cdot)$,
  sendo $\vetor{u}$ qualquer elemento de $V$, denotamos por
  \begin{enumerate}[(i)]
    \item $\simb[elemento neutro de uma operação denotada por
    $\bfmais$]{\bfzero}$ o elemento neutro de $\bfmais$, chamado de
    \conceito{vetor nulo}, e
    \item $\simb[simétrico de $\vetor{u}$ em relação a uma
      operação denotada por $\bfmais$]{\bfmenos \vetor{u}}$ o elemento
      simétrico de
      $\vetor{u}$ em relação a
    $\bfmais$;
  \end{enumerate}
\end{Not}

\begin{Obs}
  Num espaço vetorial $(V,\bfmais,\bfvezes)$ sobre um corpo
  $(F,+,\cdot)$, podemos
  sempre assumir a precedência de $\bfvezes$ em relação a
  $\bfmais$. É também
  costume escrever simplesmente $\simb[$a\bfvezes
  \vetor{u}$]{a\vetor{u}}$
  ao invés de $a\bfvezes \vetor{u}$, assim
  como $\simb[$\vetor{u}\bfmais(\bfmenos \vetor{v})$]{\vetor{u}\bfmenos
  \vetor{v}}$ ao invés de
  $\vetor{u}\bfmais(\bfmenos \vetor{v})$.
\end{Obs}

\begin{Propr}
  Num espaço vetorial $(V,\bfmais,\bfvezes)$ sobre um corpo
  $(F,+,\cdot)$, $\zero\bfvezes\vetor{u}=\bfzero$ para todo
  $\vetor{u}\in V$.
\end{Propr}

\begin{dem}
  Notemos que
  \begin{equation*}
    \begin{aligned}
      \zero\bfvezes\vetor{u} &= (\zero+\zero)\bfvezes\vetor{u} \\
        &= (\zero\bfvezes\vetor{u})\bfmais(\zero\bfvezes\vetor{u})\MMv
    \end{aligned}
  \end{equation*}
  e, portanto, que $\zero\bfvezes\vetor{u}$ é elemento neutro para a
  operação $\bfmais$. Mas, como $(V,\bfmais)$ é um grupo, o
  elemento neutro de $\bfmais$ é único
  (propriedade\xspace\ref{proprunicoe}). Logo,
  $\zero\bfvezes\vetor{u}=\bfzero$.
\end{dem}

\begin{Def}
  Sendo $(V,\bfmais,\bfvezes)$ um espaço vetorial sobre um corpo
  $(F,+,\cdot)$, diz-se que um subconjunto $S$ de $V$ é
  \Conceito{linearmente dependente}{conjunto linearmente dependente}
  sobre $(V,\bfmais,\bfvezes)$ se
  e só se existe um subconjunto finito
  $U=\cjpp{\vetor{u_k}}{k\in[\cardi{U}]}$
  de $S$ e uma função $\funcao{f}{U}{F\setminus{\zero}}$ tais que
  \begin{equation*}
    \mathbf{\sum}_{k=1}^{\cardi{U}}f(\vetor{u_k})\vetor{u_k} =
    \bfzero\MMp
  \end{equation*}
  Caso contrário, diz-se que $S$ é
  \Conceito{linearmente independente}{conjunto linearmente
  independente} sobre $(V,\bfmais,\bfvezes)$..
\end{Def}

\begin{Def}
  Sendo $(V,\bfmais,\bfvezes)$ um espaço vetorial sobre um corpo
  $(F,+,\cdot)$, diz-se que um subconjunto $S$ de $V$
  \Conceito{gera}{espaço vetorial gerado por um subconjunto do
  conjunto de vetores} o espaço vetorial $(V,\bfmais,\bfvezes)$ se e só
  se, para todo $\vetor{u}\in V$,
  existe uma função $\funcao{f_u}{S}{F}$ tal que
  \begin{equation*}
    \vetor{u} = \mathbf{\sum}_{\vetor{s}\in S} f_u(s)\vetor{s}\MMp
  \end{equation*}
  Nesse caso, dizemos que
  $f_u$ é uma \Conceito{representação}{representação
  de um vetor por uma base} de $\vetor{u}$ pela base $S$.
\end{Def}

\begin{Def}
  Sendo $(V,\bfmais,\bfvezes)$ um espaço vetorial sobre um corpo
  $(F,+,\cdot)$, diz-se que um subconjunto $B$ de $V$ é uma
  \Conceito{base}{base de um espaço vetorial} de $(V,\bfmais,\bfvezes)$
  se e só se $B$ é um conjunto linearmente independente e gera
  $(V,\bfmais,\bfvezes)$.
\end{Def}

\begin{Obs}
  Um resultado muito conhecido em Álgebra Linear, que não mostraremos no
  presente trabalho, é que quaisquer duas bases de um espaço vetorial
  correspondem-se bi\-u\-ni\-vo\-ca\-men\-te, o que nos permite a
  nomenclatura\xspace\ref{nomdimensao}.
\end{Obs}

\begin{Nom}\label{nomdimensao}
  Dizemos que um espaço vetorial $(V,\bfmais,\bfvezes)$
  sobre um corpo
  $(F,+,\cdot)$ possui \Conceito{dimensão finita}{espaço vetorial de
  dimensão finita}, ou que $(V,\bfmais,\bfvezes)$ é
  \Conceito{finito-dimensional}{espaço vetorial finito-dimensional}, se
  as bases de $(V,\bfmais,\bfvezes)$ são conjuntos finitos, ocasião em
  que chamamos a cardinalidade de qualquer uma das bases de
  \Conceito{dimensão}{dimensão de um espaço vetorial} de
  $(V,\bfmais,\bfvezes)$, denotada por $\simb[dimensão do espaço
  vetorial $(V,\bfmais,\bfvezes)$]{\dim(V,\bfmais,\bfvezes)}$.
  Se, por outro lado, as
  bases de $(V,\bfmais,\bfvezes)$ forem conjuntos infinitos, diremos que
  $(V,\bfmais,\bfvezes)$ possui \Conceito{dimensão infinita}{espaço
  vetorial de
  dimensão infinita}, ou que $(V,\bfmais,\bfvezes)$ é
  \Conceito{infinito-dimensional}{espaço vetorial
    infinito-dimensional}, e escreveremos
  $\dim(V,\bfmais,\bfvezes)=\infty$.
\end{Nom}

\section{Espaços vetoriais gerados por subcorpos}

Agora que apresentamos uma brevíssima conceituação acerca de espaços
vetoriais, dedicaremo-nos a nossa aplicação do assunto: estudaremos os
espaços vetoriais gerados por subcorpos, que nos permitirão o
desenvolvimento de algumas argumentações importantes.

\begin{Propr}
  Sendo $K$ um subcorpo de $(F,+,\cdot)$, $(F,+,\bfvezes)$ é um espaço
  vetorial sobre $(K,+_K,\cdot_K)$, dito o espaço vetorial
  gerado
  \Conceito{gerado}{espaço vetorial gerado por um subcorpo}
  por $K$, ou o espaço vetorial de $F$ sobre $K$,
  sendo $\funcao{\bfvezes}{K\times
  F}{F}$ a função definida, para todo $k\in K$ e todo $r\in F$, por:
  \begin{equation*}
    k\bfvezes \vetor{r} = k\cdot\vetor{r}\MMp
  \end{equation*}
\end{Propr}

\begin{dem}
  É imediato que $(F,+)$ seja um grupo abeliano já que $(F,+,\cdot)$ se
  trata de um
  corpo. Também é verdade, para todo $\vetor{r_1}$ e todo $\vetor{r_2}$
  em $F$, e para todo $k_1$ e todo $k_2$ em $K$, que:
  \begin{enumerate}[(i)]
    \item $\um\bfvezes\vetor{r_1} = \um\cdot\vetor{r_1} = \vetor{r_1}$;
    \item $k_1\bfvezes(k_2\bfvezes \vetor{r_1}) =
      k_1\cdot(k_2\cdot \vetor{r_1}) = (k_1\cdot k_2)\cdot\vetor{r_1} =
      (k_1\cdot_K k_2)\bfvezes \vetor{r_1}$;
    \item $k_1\bfvezes(\vetor{r_1}+\vetor{r_2}) =
      k_1\cdot(\vetor{r_1}+\vetor{r_2}) =
      (k_1\cdot\vetor{r_1})+(k_1\cdot\vetor{r_2}) =
      (k_1\bfvezes\vetor{r_1})+(k_1\bfvezes\vetor{r_2})$.
    \item $(k_1+_Kk_2)\bfvezes\vetor{r_1} =
      (k_1+k_2)\cdot\vetor{r_1} =
      (k_1\cdot\vetor{r_1})+(k_2\cdot\vetor{r_1}) =
      (k_1\bfvezes\vetor{r_1})+_K(k_2\bfvezes\vetor{r_1})$.
  \end{enumerate}
  Portanto, $(F,+,\cdot)$ é um espaço vetorial sobre $(K,+_K,\cdot_K)$.
\end{dem}

\begin{Not}
  Sendo $K$ um subcorpo de um corpo finito $(F,+,\cdot)$,
  usamos $\simb[dimensão do espaço vetorial de $F$ sobre $K$]{[F:K]}$
  para denotar a
  dimensão do espaço vetorial de $F$ sobre $K$.
\end{Not}

\begin{Teo}\label{teomkmllm}
  Sendo $L$ um subcorpo de um corpo finito
  $(M,+,\cdot)$ e $K$ um subcorpo de
  $(L,+_L,\cdot_L)$,
  \begin{equation*}
    [M:K] = [M:L][L:M]
  \end{equation*}
\end{Teo}

\begin{dem}
  Como $M$ é um corpo finito, sejam
  $A=\cjpp{\alpha_j}{j\in[[M:L]]}$
  uma base para o espaço vetorial de $M$ sobre $L$ e
  $B=\cjpp{\beta_j}{j\in[[L:K]]}$
  uma base para o espaço vetorial de $L$ sobre $K$. Assim, para todo
  $\alpha\in M$ existe uma função $\gamma$ de $[[M:L]]$ em $L$ tal que
  \begin{equation*}
    \alpha = \mathbf{\sum}_{j=1}^{[M:L]}\gamma_j\alpha_j\MMv
  \end{equation*}
  denotando-se $\gamma(j)$ por $\gamma_j$. Como o contradomínio de
  $\gamma$ é $L$ e $B$ é uma base para o espaço vetorial
  de $L$ sobre $K$, existe, para cada $j\in[[M:L]]$, uma função $r_j$ de
  $[[L:K]]$ em $K$ tal que
  \begin{equation*}
    \gamma_j
    = \mathbf{\sum}_{\ell=1}^{[L:K]}r_{(j,\ell)}\beta_{\ell}\MMv
  \end{equation*}
  denotando-se $r_j(\ell)$ por $r_{(j,\ell)}$.
  Portanto,
  \begin{equation*}
    \begin{aligned}
      \alpha &= \mathbf{\sum}_{j=1}^{[M:L]}\biggl(
      \mathbf{\sum}_{\ell=1}^{[L:K]}r_{(j,\ell)}\beta_{\ell}
      \biggr)
      \alpha_j\\
      &= \mathbf{\sum}_{j=1}^{[M:L]}
      \mathbf{\sum}_{\ell=1}^{[L:K]}
      r_{(j,\ell)}\beta_{\ell}
      \alpha_j\MMv
    \end{aligned}
  \end{equation*}
  Logo, Para mostrarmos que $[M:K]=[M:L][L:K]$, basta que mostremos que
  o conjunto $\cjpp{\beta_{\ell}\alpha_{j}}{j\in[[M:L]] e
    \ell\in[[L:K]]}$ é linearmente independente.
  Para tanto, tomemos,
  para cada $j\in[[M:L]]$ e cada $\ell\in[[L:K]]$,
    um $s_{(j,\ell)}$ tal
  que
  \begin{equation*}
    \mathbf{\sum}_{j=1}^{[M:L]}\mathbf{\sum}_{\ell=1}^{[L:K]}
    s_{(j,\ell)}\beta_{\ell}\alpha_j=\zero\MMp
  \end{equation*}
  Dessarte,
  \begin{equation*}
    \mathbf{\sum}_{j=1}^{[M:L]}\biggl(
    \mathbf{\sum}_{\ell=1}^{[L:K]}
    s_{(j,\ell)}\beta_{\ell}\biggr)\alpha_j=\zero\MMv
  \end{equation*}
  e, como $A$ é linearmente independente sobre
  o espaço vetorial de $M$ sobre $L$,
  \begin{equation*}
    \mathbf{\sum}_{\ell=1}^{[L:K]}
    s_{(j,\ell)}\beta_{\ell} = \zero
  \end{equation*}
  para todo $j\in[[M:L]]$
  e, como $B$ é linearmente independente sobre
  o espaço vetorial de $L$ sobre $K$,
  \begin{equation*}
    s_{(j,\ell)} = \zero
  \end{equation*}
  para todo $j\in[[M:L]]$ e todo $\ell\in[[L:K]]$, como queríamos
  mostrar.
\end{dem}

\begin{Teo}\label{teocardifqm}
  Se $(F,+,\cdot)$ é um corpo finito, então $\cardi{F}=q^m$ para algum
  inteiro positivo $m$, sendo $q$ a característica de $(F,+,\cdot)$.
\end{Teo}

\begin{dem}
  Vamos mostrar que existe um inteiro positivo $m$ tal que
  $F\simeq\galois{q}^{[m]}$. Para tanto, tomemos um
  subcorpo primo $K$ de $(F,+,\cdot)$. Sabemos, do
  teorema\xspace\ref{teocorposimeqgalois}, que $K\simeq
  \galois{q}$. Tomemos $m=[F:K]$ e uma base $B$ para o espaço vetorial
  de $F$ sobre $K$.
  Mostraremos, então, que $F\simeq K^B$.
  Para isso,
  basta que tomemos
  a bijeção $\funcao{f}{F}{K^B}$
  definida por:
  \begin{equation*}
    f(r) = \alpha_r\MMv
  \end{equation*}
  sendo $\funcao{\alpha_r}{B}{K}$ a representação de $r$ por $B$.
\end{dem}

\begin{Obs}\label{obsgalois}
  Podemos estender a definição de corpo de Galois para potências de
  primos da seguinte maneira: sendo $p$ um número primo, $m$ um inteiro
  positivo
  e $q=p^m$, utilizamos $\simb[corpo de Galois de ordem
  $q=p^m$]{\galois{q}}$ para denotarmos o conjunto
  $[0..(q-1)]$. É óbvio que $(\galois{q},+,\cdot)$ também
  é um corpo, chamado
  de \conceito{corpo de Galois} de \Conceito{ordem}{ordem de um corpo de
  Galois} $q$, sendo as operações $+$ e $\cdot$ módulo $q$.
\end{Obs}

\begin{Propr}\label{proprgaloismn}
  Se $m$ é um divisor positivo de um inteiro positivo
  $n$ e $p$ é um primo positivo
  então $\galois{p^m}$ é um subcorpo de $(\galois{p^n},+,\cdot)$.
\end{Propr}

\begin{dem}
  Como $m$ divide $n$, $\galois{p^m}$ é evidentemente um subconjunto de
  $\galois{p^n}$. Da observação\xspace\ref{obsgalois},
  $(\galois{p^m},+,\cdot)$ é um corpo.
\end{dem}

%%%%%%%%%%%%%%%%%%%%%%%%%%%%%%%%%%%%%%%%%%%%%%%%%%%%%%%%%%%%%%%%%%%%%%%%
% estralg.tex                                                          %
%                                                       fim do arquivo %
%%%%%%%%%%%%%%%%%%%%%%%%%%%%%%%%%%%%%%%%%%%%%%%%%%%%%%%%%%%%%%%%%%%%%%%%
